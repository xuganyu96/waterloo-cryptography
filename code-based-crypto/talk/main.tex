% TODO: outline of the talk:
% - Overview: what is Classic McEliece, summary of Goppa code and the cryptosystem
% - Preliminaries: finite field, polynomial ring, Bernoulli's rule
% - Goppa decoder:
%   - Lagrange interpolation
%   - Best approximant theorem
%   - Interpolate with error as a Reed Solomon decoder
%   - Definition of code space, Goppa squaring theorem
%   - Goppa decoder via Reed-Solomon decoder
% - Canonical parity check matrix, systematic form, syndrome decoding
% - Implementation details
%   - Semi-systematic form
\documentclass{article}
\usepackage[margin=1in,letterpaper]{geometry}
\usepackage{amsmath,amsfonts,amssymb,amsthm}

% For source code
\usepackage{listings}

% Algorithms and pseudocode
% \usepackage[linesnumbered,ruled,vlined]{algorithm2e}
\usepackage{algorithm}
\usepackage{algpseudocode}
\usepackage{hyperref}

% Custom commands
\usepackage{crypto-primitives}

% Environments: definitions, theorems, propositions, corollaries, lemmas
%    Theorems, propositions, and definitions are numbered within the section
%    Corollaries are numbered within the theorem, though they are rarely used
\newtheorem{definition}{Definition}[section]
\newtheorem{theorem}{Theorem}[section]
\newtheorem*{remark}{Remark}
\newtheorem{corollary}{Corollary}[theorem]
\newtheorem{proposition}{Proposition}[section]
\newtheorem{lemma}{Lemma}[theorem]


\title{
    A graduate introduction to classic McEliece
}
\author{
    Ganyu (Bruce) Xu
}
% Leave the date field empty to display the date of compilation
\date{November, 2024}

\begin{document}
%%%% TITLE %%%%%
\maketitle

\section{Introduction}
Classic McEliece is an IND-CCA2 key encapsulation mechanism (KEM) submitted to NIST's post-quantum cryptography (PQC) standardization project. Its security is based on the conjectured intractability of the \textbf{Syndrome Decoding Problem}.

This document provides an introduction to the mathematics behind classic McEliece with a particular emphasis on the details of binary Goppa code. We assume a graduate level of mathematical maturity, though most of the results can be reasoned about using first or second year undergraduate math.

Classic McEliece is an appealing candidate for PQC because its cryptanalysis has been remarkably stable since its earliest conception in 1978 \cite{mceliece1978public} (whereas lattice-based cryptography saw enormous advance in its cryptanalysis in the last decade alone, forcing many latticed-based schemes to revise their parameters). Classic McEliece also has the following advantages: \begin{itemize}
    \item Encryption is faster
    \item Ciphertext is smaller
    \item There is no decryption failure that can leak information about secret key
\end{itemize}

On the other hand, Classic McEliece has larger public key size and slower decryption routine.

\subsection{A summary of binary Goppa code}
Binary Goppa code is a linear error-correcting code. Each instance is parameteried by: \begin{itemize}
    \item some base field $K = \mathbb{F}_{2^m}$
    \item $n$ distinct field elements $\alpha_1, \alpha_2, \ldots, \alpha_n \in K$
    \item An irreducible polynomial $g \in K[x]$ such that $\deg(g) = t$
\end{itemize} 

% TODO: need better summary

\subsection{An overview of Classic McEliece}
Each instance of Classic McEliece is parameterized by three integers: \begin{itemize}
    \item $m$ is the size of the base field $K = \mathbb{F}_{2^m}$
    \item $n$ is the size of code words: $\text{codewords} \subseteq K^n$
    \item $t$ is the error-correcting capacity of the underlying Goppa code
\end{itemize}

\paragraph{Key generation.} Begin by randomly generating the parameters of a Goppa code instance, which include: \begin{itemize}
    \item $n$ distinct field elements $\alpha_1, \alpha_2, \ldots, \alpha_n \in K$
    \item A degree-$t$, square-free polynomial $g \in K[x]$ such that $g(\alpha_i) \neq 0$ for $1 \leq i \leq n$
\end{itemize}

$\alpha_1, \alpha_2, \ldots, \alpha_n$ and $g$ can be used to compute the canonical parity-check matrix $H \in K^{t \times n}$

$$
H_{i,j} = \frac{\alpha_j^{i - 1}}{g(\alpha_j)} \text{ for } 1 \leq i \leq t, 1 \leq j \leq n
$$

$H$ is then transformed into reduced row-echelon form (or systematic form per DJB) $H^\prime = [I_t \mid T]$ for some $T \in K^{t \times (n - t)}$. $T$ is returned as the public key. $\Gamma = (\alpha_1, \alpha_2, \ldots, \alpha_n, g)$ is returned as the secret key.

\paragraph{Encryption.} The message space is the subset of $\mathbb{F}_2^n$ whose Hamming weight is exactly $t$:

$$\mathcal{M} = \{\mathbf{e} \in \mathbb{F}_2^n \mid \mathop{wt}(\mathbf{e}) = t\}$$

To encrypt, compute:

$$\mathbf{y} \leftarrow [I_t \mid T] \mathbf{e}$$

$\mathbf{y} \in K^t$ is returned as the ciphertext.

\paragraph{Decryption.} Given $\mathbf{y} \in K^t$ and $\Gamma = (\alpha_1, \alpha_2, \ldots, \alpha_n, g)$, let $\mathbf{r} \leftarrow (y_1, y_2, \ldots, y_t, 0, 0, \ldots, 0) \in \mathbb{F}_2^n$. then feed $(\Gamma, \mathbf{r})$ into some Goppa decoder, which will directly recover $\mathbf{e} \in \mathbb{F}_2^n$ such that $\mathop{wt}(\mathbf{e}) = t$. Return $\mathbf{e}$ as the decryption.

\section{Implementation details}
% TODO: semi-systematic form
% TODO: how to generate field ordering and irreducible polynomial

\bibliographystyle{alpha}
\bibliography{references}

\end{document}