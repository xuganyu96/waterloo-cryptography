\documentclass{article}
\usepackage[margin=1in,letterpaper]{geometry}
\usepackage{amsmath,amsfonts,amssymb,amsthm}

% For code
\usepackage{listings}


\newcommand{\round}[1]{\lfloor {#1} \rceil}
\newcommand{\norm}[1]{\Vert {#1} \Vert}
\newcommand{\var}[1]{\operatorname{Var}[{#1}]}

\title{CO 789, Homework 1}
\author{Ganyu (Bruce) Xu (g66xu)}
\date{Fall 2023}

\begin{document}
% Title is not required when submitting to Crowdmark
% \maketitle

\section*{Question 2}

\subsection*{(1)}
Recall in the IND-CPA security proof, we defined three games:

\begin{enumerate}
    \item Game 0 is the standard IND-CPA game for $\operatorname{Module-LWE}$
    \item Game 1 is identical to game 0, except in key generation, $\mathbf{b} = A\mathbf{s} + \mathbf{e}$ is replaced with a uniformly random sample $\mathbf{b} \leftarrow R_q^k$
    \item Game 2 is identical to game 1, except the challenge ciphertexts are replaced with uniformly random samples $\mathbf{c}_1^\ast \leftarrow R_q^k, c_2^\ast \leftarrow R_q$
\end{enumerate}

Further more, we defined two solvers for Module-decisional-LWE: solver 1 solves dLWE with $A \in R_q^{k \times k}$ and solver 2 solves dLWE with $A \in R_q^{(k+1) \times k}$.

Solver 2 decomposes $A, \mathbf{b}$ into the first $k$ rows and the last row: $A = [A_1 \in R_q^{k \times k},  A_2 \in R_q^{1 \times k}]$, $\mathbf{b} = [\mathbf{b}_1\in R_q^{k}, b_2 \in R_q]$. $A_1, A_2$ is given to the IND-CPA adversary as the public key, and $\mathbf{c}_1^\ast = \mathbf{b}_1, c_2^\ast = b_2 + m \round{\frac{q}{2}}$ as the challenge ciphertext. If Solver 2 receives LWE sample, then the IND-CPA adversary is playing game 1; if solver 2 receives truly random sample, then IND-CPA adversary is playing game 2. Therefore, the advantage of solver 2 is $\frac{1}{2}(\operatorname{adv}_1 - \operatorname{adv}_2)$

If in the encryption routine, the second error term $e^{\prime\prime}$ is removed, then $c_2^\ast = b_2 + m\round{\frac{q}{2}} = A_2\mathbf{s} + e_2 + m\round{\frac{q}{2}}$ is no longer a valid encryption of $m$. This means that when $A, \mathbf{b}$ is a LWE sample, the IND-CPA adversary is not playing game 1, but a new game that is identical to game 1 but with the second error term in the encryption routine. Denote the IND-CPA adversary's advantage in this game by $\operatorname{adv}_3$.

Following the same procedure as in the IND-CPA security proof, we can show that:

$$
\text{Adv in solving dLWE($R_q^{k \times k}$)} + 
\text{Adv in solving dLWE($R_q^{k+1 \times k}$)}
= \frac{1}{2}(\operatorname{adv}_0 - \operatorname{adv}_1) 
+ \frac{1}{2}(\operatorname{adv}_3 - \operatorname{adv}_2)
$$

Knowing that game 2 is unwinnable and that solving dLWE with higher dimension is harder, we can rearrange the equation above:

$$
\operatorname{adv}_0 
\geq 4 \cdot \operatorname{adv}_\text{dLWE(k)} + (\operatorname{adv}_1 - \operatorname{adv}_3)
$$

It's possible that $\operatorname{adv}_1 - \operatorname{adv}_3$ is non-negligible, so $\operatorname{adv}_0$ might be non-negligible, thus breaking IND-CPA security of the modified encryption routine.

\subsection*{(2)}
Observe that when $m = \mathbf{0} \in R_{\{0,1\}}$, $c_2 = \mathbf{r}^\intercal \mathbf{b}$ is divisible by $\mathbf{b}$. On the other hand, if $m$ is non-zero, then with very high probability, $c_2$ will not be divisible by $\mathbf{b}$.

Thus, an IND-CPA can challenge plaintexts as follows: $m_0 = \mathbf{0}$, $m_1$ is some arbitrary non-zero polynomial. Upon receiving the challenge ciphertext $c^\ast$, check if $c_2$ is divisible by $\mathbf{b}$ (which is part of the public key and hence accessible). If yes, then $c^\ast$ is the encryption of $m_0$; otherwise $c^\ast$ is the encryption of $m_1$.

\end{document}