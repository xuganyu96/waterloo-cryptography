\documentclass{article}
\usepackage[margin=1in,letterpaper]{geometry}
\usepackage{amsmath,amsfonts,amssymb,amsthm}

% For code
\usepackage{listings}


\newcommand{\round}[1]{\lfloor {#1} \rceil}
\newcommand{\norm}[1]{\Vert {#1} \Vert}
\newcommand{\var}[1]{\operatorname{Var}[{#1}]}
\newcommand{\indcpa}{\mathcal{A}_\text{IND-CPA}}
\newcommand{\indcca}{\mathcal{A}_\text{IND-CCA}}

\title{CO 789, Homework 1}
\author{Ganyu (Bruce) Xu (g66xu)}
\date{Fall 2023}

\begin{document}
% Title is not required when submitting to Crowdmark
% \maketitle

\section*{Question 1}

We show that IND-CCA security implies IND-CPA security by showing that if an IND-CPA adversary can win with non-negligible advantage, then it can be used to build an IND-CCA adversary who can win with non-negligible advantage.

The key generation routines are identical between the IND-CPA and the IND-CCA game. When $\indcca$ receives the keypair, it directly passes it to $\indcpa$. $\indcpa$ can use the public key to perform encryption queries but does not submit any decryption queries.

When $\indcpa$ generates the challenge plaintext $m_0, m_1$, $\indcca$ passes them to the IND-CCA challenger and receives the challenge ciphertext $c_b$. $\indcca$ passes $c_b$ to $\indcpa$ and receives the guess $b^\ast$ from $\indcpa$.

Because both the IND-CPA and the IND-CCA games are played with identical keypairs, $\indcpa$'s guess is correct if and only if $b^\ast$ is equal to $b$. Therefore, $\indcca$'s advantage is equal to $\indcpa$'s advantage, meaning that if $\indcpa$ has non-negligible advantage, then $\indcca$ has non-negligble advantage.

\end{document}