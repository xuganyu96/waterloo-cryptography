\documentclass{article}
\usepackage[margin=1in,letterpaper]{geometry}
\usepackage{amsmath,amsfonts,amssymb,amsthm}

% For code
\usepackage{listings}


\newcommand{\round}[1]{\lfloor {#1} \rceil}
\newcommand{\norm}[1]{\Vert {#1} \Vert}
\newcommand{\var}[1]{\operatorname{Var}[{#1}]}

\title{CO 789, Homework 1}
\author{Ganyu (Bruce) Xu (g66xu)}
\date{Fall 2023}

\begin{document}
% Title is not required when submitting to Crowdmark
% \maketitle

\section*{Question 3}

Without rounding $\mathbf{w}$, the public key is $(A, \mathbf{t} = A\mathbf{s} + \mathbf{e})$ the signature contains three components:

$$
\begin{aligned}
    \mathbf{z} = \mathbf{y} + c \mathbf{s} \\
    c = H(\mathbf{w}, m) \\
    \mathbf{w} = A\mathbf{y}
\end{aligned}
$$

Multiply $\mathbf{z}$ by $A$:

$$
\begin{aligned}
    A\mathbf{z} &= A\mathbf{y} + c A\mathbf{s} \\
    &= \mathbf{w} + c \cdot (\mathbf{t} - \mathbf{e})
\end{aligned}
$$

Assuming that $c$ is invertible, re-arranging the equation above allows us to recover $\mathbf{e}$ from the publis key $A, \mathbf{t}$ and a single pair of message and signature $(m, (\mathbf{w}, c, \mathbf{z}))$:

$$
\mathbf{e} = \mathbf{t} - c^{-1}(A\mathbf{z} - \mathbf{w})
$$

From here we can attempt to recover the secret key $\mathbf{s}$ by solving $A\mathbf{s} = \mathbf{t} - \mathbf{e}$. While this is a non-trivial instance of inhomogeneous SIS, with proto-Dilithium $A$ is a wide matrix, which makes SIS easier to solve. In the lecture notes, we simply assume that for the choice of parameters in proto-Dilihtium, such wide ISIS is "too easy".

\end{document}