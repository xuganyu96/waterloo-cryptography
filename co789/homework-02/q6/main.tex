\documentclass{article}
\usepackage[margin=1in,letterpaper]{geometry}
\usepackage{amsmath,amsfonts,amssymb,amsthm}

% For code
\usepackage{listings}


\newcommand{\round}[1]{\lfloor {#1} \rceil}
\newcommand{\norm}[1]{\Vert {#1} \Vert}
\newcommand{\var}[1]{\operatorname{Var}[{#1}]}

\title{CO 789, Homework 1}
\author{Ganyu (Bruce) Xu (g66xu)}
\date{Fall 2023}

\begin{document}
% Title is not required when submitting to Crowdmark
% \maketitle

\section*{Question 6}
\subsection*{(a)}
We denote a polynomial with degree $\phi(n) - 1$ by $f(x) = f_0 + f_1x + \ldots + f_{\phi(n)-1}x^{\phi(n)-1}$. It is trivially true that if $f(x) = g(x)$, then $f(\zeta_i) = g(\zeta_i)$ for all $1 \leq i \leq \phi(n)$.

On the other hand, if for two polynomials $f, g$ with degree $\phi(n) - 1$, their NTT representations are exactly identical, then for all $1 \leq i \leq \phi(n)$:

$$
f(\zeta_i) = g(\zeta_i)
$$

Define $h(x) = f(x) - g(x)$, then $\zeta_1, \zeta_2, \ldots, \zeta_{\phi(n)}$ are distinct roots of $h(x)$. This means that $h(x)$ must have form:

$$
h(x) = h^\prime(x)(x - \zeta_1)(x - \zeta_2) \ldots (x - \zeta_{\phi(n)})
$$

Because $f(x), g(x)$ both have degree $\phi(n) - 1$, the degree of $f(x) - g(x)$ cannot be more than $\phi(n) - 1$. Therefore, $h(x)$ must be $0$, because otherwise it must have degree of at least $\phi(n)$. $\blacksquare$

\subsection*{(b)}
Recall the NTT transformation described in part (1): $\operatorname{NTT}(f) \mapsto (f(\zeta_1), f(\zeta_2), \ldots, f(\zeta_{\phi(n)}))$. From here we know:

$$
\begin{aligned}
\operatorname{NTT}(f \cdot g) 
&= ((f \cdot g)(\zeta_1), (f \cdot g)(\zeta_2), \ldots, (f \cdot g)(\zeta_{\phi(n)})) \\
&= (f(\zeta_1) \cdot g(\zeta_1), f(\zeta_2) \cdot g(\zeta_2), \ldots, f(\zeta_{\phi(n)}) \cdot g(\zeta_{\phi(n)})) \\
&= (f(\zeta_1), f(\zeta_2), \ldots, f(\zeta_{\phi(n)})) \circ (g(\zeta_1), g(\zeta_2), \ldots, g(\zeta_{\phi(n)})) \\
&= \operatorname{NTT}(f) \circ \operatorname{NTT}(g)
\end{aligned}
$$

$\blacksquare$

\end{document}