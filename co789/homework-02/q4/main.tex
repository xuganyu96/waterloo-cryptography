\documentclass{article}
\usepackage[margin=1in,letterpaper]{geometry}
\usepackage{amsmath,amsfonts,amssymb,amsthm}

% For code
\usepackage{listings}


\newcommand{\round}[1]{\lfloor {#1} \rceil}
\newcommand{\norm}[1]{\Vert {#1} \Vert}
\newcommand{\var}[1]{\operatorname{Var}[{#1}]}

\title{CO 789, Homework 1}
\author{Ganyu (Bruce) Xu (g66xu)}
\date{Fall 2023}

\begin{document}
% Title is not required when submitting to Crowdmark
% \maketitle

\section*{Question 4}

\subsection*{(a)}
By the definition of the (quotient) ring $R_q = \mathbb{Z}_q[x] / \langle p(x) \rangle$ we know that:

\begin{equation}
    a(x)s(x) + e(x) = p(x)g(x) + b(x)
\end{equation}

Where $g(x)$ is some polynomial in $\mathbb{Z}_q[x]$.

Where $\omega$ is a root of $p(x)$, evaluating equation (1) at $x = \omega$ is as follows:

$$
a(\omega)s(\omega) + e(\omega) = 0 \cdot g(\omega) + b(\omega) = b(\omega)
$$

$\blacksquare$

\subsection*{(b)}
From equation (1) we know that $a(\omega)s(\omega) + e(\omega) = b(\omega)$ if and only if $p(\omega)g(\omega) = 0$. Where $\omega$ is not a root of $p(x)$, the equality holds if and only if $g(\omega) = 0$.

Because $a(x), s(x), e(x) \in R_q$ are all polynomials of degree (up to) $d-1$, and $p(x)$ is a polynomial of degree $d$, the degree of $g(x)$ cannot be more than $2 \cdot (d - 1) - d = d - 2$. By the fundamental theorem of algebra we know that $g(x)$ cannot have more than $d-2$ roots in $\mathbb{Z}_q$. Therefore, if $q > d-2$, then there exists $\omega \in \mathbb{Z}_q$ such that $\omega$ is not a root of $g(x)$, which means that $b(\omega) \neq a(\omega)s(\omega) + e(\omega)$.

On the other hand, if the sum of degrees of $a(x)$ and $s(x)$ is less than the degree of $p(x)$, then $g(x) = 0$ (aka $b(x)$ does not need to be reduced modulus $p(x)$).


\end{document}