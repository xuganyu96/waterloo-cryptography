\documentclass{article}
\usepackage[margin=1in,letterpaper]{geometry}
\usepackage{amsmath,amsfonts,amssymb,amsthm}
\usepackage{algorithm}
\usepackage{algpseudocode}

% For code
\usepackage{listings}


\newcommand{\round}[1]{\lfloor {#1} \rceil}
\newcommand{\norm}[1]{\Vert {#1} \Vert}
\newcommand{\var}[1]{\operatorname{Var}[{#1}]}

\title{CO 789, Homework 1}
\author{Ganyu (Bruce) Xu (g66xu)}
\date{Fall 2023}

\begin{document}
% Title is not required when submitting to Crowdmark
% \maketitle

\section*{Question 3}
First, the result of the multiplication in the quotient ring is as follows:

\begin{equation}
    (a_1 + a_2x)(b_1 + b_2x) \equiv (a_1b_1 + a_2b_2\zeta) + (a_1b_2 + a_2b_1)x \mod x^2 - \zeta
\end{equation}

Using schoolbook multiplication, the R.H.S. from above requires 5 multiplication. Using Karatsuba we can compute $a_1b_2 + a_2b_1$ using only one multiplication (but at the expense at more addition/subtraction):

\begin{equation}
    a_1b_2 + a_2b_1 = (a_1 + a_2)(b_1 + b_2) - a_1b_1 - a_2b_2
\end{equation}

The R.H.S. of equation (2) only takes one multiplication because $a_1b_1$ and $a_2b_2$ have already been computed from previous steps of schoolbook multiplication.

Putting everything together:

\begin{algorithm}
\caption{Karatsuba-ish monomial multiplication}\label{alg:cap}
\begin{algorithmic}
    
    \State Start with $a_1 + a_2x$ and $b_1 + b_2x$
    \State $c_1 \leftarrow a_1 b_1$ \Comment first multiplication
    \State $c_3 \leftarrow a_2b_2$ \Comment second multiplication
    \State $c_2 \leftarrow (a_1 + a_2)(b_1 + b_2) - c_1 - c_3$ \Comment third multiplication
    \State $c_3 \leftarrow c_3 \zeta$ \Comment fourth multiplication

\Return $(c_1 + c_3) + c_2x$

\end{algorithmic}
\end{algorithm}

\end{document}