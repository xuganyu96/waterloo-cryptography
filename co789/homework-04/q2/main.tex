\documentclass{article}
\usepackage[margin=1in,letterpaper]{geometry}
\usepackage{amsmath,amsfonts,amssymb,amsthm}

% For source code
\usepackage{listings}

% Algorithms and pseudocode
\usepackage{algorithm}
\usepackage{algpseudocode}

% Common shortcuts
\newcommand{\round}[1]{\lfloor {#1} \rceil}
\newcommand{\norm}[1]{\Vert {#1} \Vert}
\newcommand{\var}[1]{\operatorname{Var}[{#1}]}

% Environments: definitions, theorems, propositions, corollaries, lemmas
%    Theorems, propositions, and definitions are numbered within the section
%    Corollaries are numbered within the theorem, though they are rarely used
\newtheorem{definition}{Definition}[section]
\newtheorem{theorem}{Theorem}[section]
\newtheorem*{remark}{Remark}
\newtheorem{corollary}{Corollary}[theorem]
\newtheorem{proposition}{Proposition}[section]
\newtheorem{lemma}{Lemma}[theorem]


\title{CO 789, Homework 1}
\author{Ganyu (Bruce) Xu (g66xu)}
\date{Fall 2023}

\begin{document}
%%%% TITLE %%%%%
% \maketitle

\section*{Question 2}
In SPX, each WOTS keypair $\text{WOTS}$ signs the root of child XMSS tree(s). If WOTS is replaced by some $k$-time signature scheme, then each leaf node in each XMSS tree can sign the roots of $k$ child XMSS tress. Fixing each XMSS tree to still have $2^t$ leaf notes, then each XMSS tree can have $k \cdot 2^t$ children. If the entire SPX hypertree has $d$ layers, then it can sign a total of $(k \cdot 2^t)^d$ messages.

From the lecture notes we know that for a target security level $\lambda$, we want:

$$
(k \cdot 2^t)^d = 2^{2\lambda}
$$

Which solves to $d = \frac{2\lambda}{\log_2{k} + t}$. With larger $k$ (aka the signature scheme can sign more messages without losing security), \textbf{we need fewer layers in the hypertree to accomplish the same security level}. Since the signature, signing routine, and verification routine all iterate through all $d$ layers of the hypertree, \textbf{signature size, signing time, and verification time all decrease linearly as $d$ decreases}

\end{document}