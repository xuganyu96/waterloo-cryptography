\documentclass{article}
\usepackage[margin=1in,letterpaper]{geometry}
\usepackage{amsmath,amsfonts,amssymb,amsthm}

% For source code
\usepackage{listings}

% Algorithms and pseudocode
\usepackage{algorithm}
\usepackage{algpseudocode}

% Common shortcuts
\newcommand{\round}[1]{\lfloor {#1} \rceil}
\newcommand{\norm}[1]{\Vert {#1} \Vert}
\newcommand{\var}[1]{\operatorname{Var}[{#1}]}

% Environments: definitions, theorems, propositions, corollaries, lemmas
%    Theorems, propositions, and definitions are numbered within the section
%    Corollaries are numbered within the theorem, though they are rarely used
\newtheorem{definition}{Definition}[section]
\newtheorem{theorem}{Theorem}[section]
\newtheorem*{remark}{Remark}
\newtheorem{corollary}{Corollary}[theorem]
\newtheorem{proposition}{Proposition}[section]
\newtheorem{lemma}{Lemma}[theorem]


\title{CO 789, Homework 1}
\author{Ganyu (Bruce) Xu (g66xu)}
\date{Fall 2023}

\begin{document}
%%%% TITLE %%%%%
% \maketitle

\section*{Question 3}
Recall that the FORS scheme uses $k$ Merkle trees $T_1, T_2, \ldots, T_k$ each with $n$ leaf nodes. At signing $m$ is hashed into $1 \leq h_1, h_2, \ldots, h_k \leq n$, and the signature are the authentication paths $\sigma_j$ of $h_j$ in tree $T_j$ for $1 \leq j \leq k$. 

Let $\mathcal{A}$ denote a EF-CMA adversary. Suppose that $\mathcal{A}$ makes $N$ queries: $m_1, m_2, \ldots, m_N$, then the signing oracles will return the authentication paths for:

$$
\begin{bmatrix}
    h_{1, 1}, &h_{1, 2}, &\ldots, &h_{1, N}, &\text{from } T_1 \\
    h_{2, 1}, &h_{2, 2}, &\ldots, &h_{2, N}, &\text{from } T_2 \\
    \ldots \\
    h_{k, 1}, &h_{k, 2}, &\ldots, &h_{k, N}, &\text{from } T_k \\
\end{bmatrix}
$$

Let $m^\ast$ be some randomly chosen message, and $h_1^\ast, h_2^\ast, \ldots, h_k^\ast = H(m)$. $m^\ast$ is forgeable if the authentication paths in all $k$ trees are forgeable. An authentication path in tree $T_i$ is forgeable if any of $h_{i,1}, h_{i,2}, \ldots, h_{i, N}$ collides with $h_i^\ast$. Therefore:

$$
\begin{aligned}
    P[\text{$T_i$ auth path forgeable}] 
    &= 1 - P[\text{$T_i$ auth path unforgeable}] \\
    &= 1 - P[h_{i,j} \neq h_i^\ast \text{ for all } 1 \leq j \leq N] \\
    &= 1 - \prod_{j=1}^N P[h_{i,j} \neq h_i^\ast] \\
    &= 1 - (1 - \frac{1}{n})^N
\end{aligned}
$$

Where $P[h_{i,j} \neq h_i^\ast] = 1 - \frac{1}{n}$ because each hash $H(m)$ must be in the range $1, 2, \ldots, n$, and we assume that hash function to be an ideal pseudorandom function.

Finally:

$$
\begin{aligned}
P[\text{FORS forgeable}]
&= P[\text{$T_i$  authentication path forgeable for all $1 \leq i \leq k$}] \\
&= \prod_{i=1}^k P[\text{$T_i$ auth path forgeable}] \\
&= (1 - (1 - \frac{1}{n})^N)^k
\end{aligned}
$$

\end{document}