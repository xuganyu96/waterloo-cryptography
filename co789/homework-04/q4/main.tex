\documentclass{article}
\usepackage[margin=1in,letterpaper]{geometry}
\usepackage{amsmath,amsfonts,amssymb,amsthm}

% For source code
\usepackage{listings}

% Algorithms and pseudocode
\usepackage{algorithm}
\usepackage{algpseudocode}

% Common shortcuts
\newcommand{\round}[1]{\lfloor {#1} \rceil}
\newcommand{\norm}[1]{\Vert {#1} \Vert}
\newcommand{\var}[1]{\operatorname{Var}[{#1}]}

% Environments: definitions, theorems, propositions, corollaries, lemmas
%    Theorems, propositions, and definitions are numbered within the section
%    Corollaries are numbered within the theorem, though they are rarely used
\newtheorem{definition}{Definition}[section]
\newtheorem{theorem}{Theorem}[section]
\newtheorem*{remark}{Remark}
\newtheorem{corollary}{Corollary}[theorem]
\newtheorem{proposition}{Proposition}[section]
\newtheorem{lemma}{Lemma}[theorem]


\title{CO 789, Homework 1}
\author{Ganyu (Bruce) Xu (g66xu)}
\date{Fall 2023}

\begin{document}
%%%% TITLE %%%%%
% \maketitle

\section*{Question 4}
With a ternary Merkle tree (3MT for short), each piece of data being committed will have 2 sibling nodes at each layer in the 3MT. If the 3MT has $3^t$ leaf nodes, then the authentication paths will contain $2t$ hashes (compared to $t$ hashes in regular MT).

In the SPX hypertree, each layer produces one WOTS signature and one Merkle tree authentication path. If we use 3MT for SPX then:

$$
\vert \sigma^\text{SPX} \vert = d(
    \vert \sigma^\text{WOTS} \vert + 2t \vert H \vert
) 
+ \vert \sigma^\text{FORS} \vert
$$

where $H$ is the hash function used in the Merkle tree.

Note that if we use 3MT, then each XMSS will have $3^t$ leaf nodes, and the entire hypertree will be able to sign $3^{dt}$ distinct messages. With a fixed security level $\lambda$ and $t$, the value of $d$ will decreased, meaning that we will need fewer layers in the hypertree (so the signature size is not strictly increasing).

\end{document}