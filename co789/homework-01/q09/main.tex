\documentclass{article}
\usepackage[margin=1in,letterpaper]{geometry}
\usepackage{amsmath,amsfonts,amssymb,amsthm}

% For code
\usepackage{listings}


\newcommand{\round}[1]{\lfloor {#1} \rceil}
\newcommand{\norm}[1]{\Vert {#1} \Vert}
\newcommand{\var}[1]{\operatorname{Var}[{#1}]}

\title{CO 789, Homework 1}
\author{Ganyu (Bruce) Xu (g66xu)}
\date{Fall 2023}

\begin{document}
% Title is not required when submitting to Crowdmark
% \maketitle

\section*{Q9}
Let $(A, \mathbf{b}) \leftarrow \operatorname{LWE}(n, n, q, \chi_e, \chi_s)$. In other words, $A$ is uniformly randomly sampled from all full-rank matrices $\mathbb{F}_q^{n \times n}$, $\mathbf{s} \leftarrow \chi_e^n$, $\mathbf{e} \leftarrow \chi_s^n$, and $\mathbf{b} \leftarrow A\mathbf{s} + \mathbf{e}$.

I claim that matrix inversion $A \mapsto A^{-1}$ on the set of full-rank matrices is a bijection. This is true because the inverse of two distinct matrices is necessarily distinct (injectivity), and every full-rank matrix is the inverse of its inverse (surjectivity).

Because matrix inversion is a bijection from the set of invertible matrices onto itself, if $A$ is uniformly sampled from all full-rank matrices $\mathbb{F}_q^{n \times n}$, then $A^{-1}$ is also a uniformly randomly sampled matrix from the set all full-rank matrices.

Notice that $A^{-1}\mathbf{b} = A^{-1}A\mathbf{s} + \mathbf{e} = A^{-1}\mathbf{e} + \mathbf{s}$. Since $\mathbf{e}$ is sampled from the secret distribution and $\mathbf{s}$ is sampled from the error distribution, $A^{-1}$ is a uniformly random sample, $(A^{-1}, A^{-1}\mathbf{b}) = (A^{-1}, A^{-1}\mathbf{e} + \mathbf{s})$ is a sample from $\operatorname{LWE}(n, n, q, \chi_s, \chi_e)$. If there exists an oracle for $\operatorname{LWE}(n, n, q, \chi_s, \chi_e)$, then this oracle can recover $\mathbf{e}$, and from here we can recover $\mathbf{s}$ in $\operatorname{LWE}(n, n, q, \chi_e, \chi_s)$.

The argument above did not assume anything specific to $\chi_s$ or $\chi_e$, so they can be swapped, and we will arrive at the inverse conclusion that $\operatorname{LWE}(n, n, q, \chi_s, \chi_e)$ reduces to $\operatorname{LWE}(n, n, q, \chi_e, \chi_s)$, as well. Therefore, the two problems are equivalent.


\end{document}