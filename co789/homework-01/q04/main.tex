\documentclass{article}
\usepackage[margin=1in,letterpaper]{geometry}
\usepackage{amsmath,amsfonts,amssymb,amsthm}

% For code
\usepackage{listings}


\newcommand{\round}[1]{\lfloor {#1} \rceil}
\newcommand{\norm}[1]{\Vert {#1} \Vert}
\newcommand{\var}[1]{\operatorname{Var}[{#1}]}

\title{CO 789, Homework 1}
\author{Ganyu (Bruce) Xu (g66xu)}
\date{Fall 2023}

\begin{document}
% Title is not required when submitting to Crowdmark
% \maketitle

\section*{Q4}
Using the Kyber described in the definition sheet, the LWE parameters are as follows: $n = m$, $q = 3329$, $\chi_s = \mathcal{B}(n=6, p=0.5)$, $\chi_e = \mathcal{B}(n=4, p=0.5)$, where $\mathcal{B}(n, p)$ denotes the centered binomial distributions.

Since $\mathbf{s} \leftarrow \chi_s^{n}$ is independently sampled from identical distributions, we can describe $\norm{\mathbf{s}}^2$ as the sum of I.I.D. random variables:

$$
\norm{\mathbf{s}}^2 = \sum_{i=1}^{n} S_i^2
$$

Therefore:

$$
E[\norm{\mathbf{s}}^2] = E\bigg[\sum_{i=1}^{n} S_i^2\bigg] = \sum_{i=1}^{n}E[S_i^2]
$$

Because $S_i$ follows the \textbf{centered} binomial distribution, $E[S_i] = 0$, so $E[S_i^2] = \var{S_i}$. On the other hand, the variance of the centered binomial distribution is identical to that of the corresponding binomial distribution: $\var{S_i} = 6 \cdot p(1-p) = \frac{3}{2}$. This is true because shifting a random variable by a constant does not change its variability.

Putting everything together:

$$
E[\norm{\mathbf{s}}^2] = \sum_{i=1}^{n}E[S_i^2] = \frac{3}{2} n
$$

On the other hand, for calculating the variance of $\norm{\mathbf{s}}^2$, we take advantage of the fact that the entries of $\mathbf{s}$ are independently drawn, and the variance of sum of independent random variables is the sum of variances:

$$
\begin{aligned}
\var{\norm{s}^2} &= \var{\sum_{i=1}^n S_i^2} \\
&= \sum_{i=1}^n \var{S_i^2} \\
&= \sum_{i=1}^n (E[S_i^4] - E[S_i^2]^2) \\
&= \sum_{i=1}^n \bigg(
    (\sum_{j=-3}^3(j^4 \cdot \binom{6}{j} \cdot 2^{-6}) 
    - (\frac{3}{2})^2 
\bigg) \\
&= \sum_{i=1}^n \bigg( 6  - \frac{9}{4} \bigg) \\
&= \frac{15}{4}n
\end{aligned}
$$

Replacing the secret distribution with the error distribution, we can compute the expectation and variance of the norm square of the error term in similar fashion. In conclusion:

$$
\begin{aligned}
E[\norm{\mathbf{s}}^2] &= \frac{3}{2} n \\
\var{\norm{s}^2} &= \frac{15}{4}n \\
E[\norm{\mathbf{e}}^2] &= n \\
\var{\norm{e}^2} &= \frac{3}{2}n
\end{aligned}
$$


\end{document}