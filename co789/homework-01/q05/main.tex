\documentclass{article}
\usepackage[margin=1in,letterpaper]{geometry}
\usepackage{amsmath,amsfonts,amssymb,amsthm}

% For code
\usepackage{listings}

\title{CO 789, Homework 1}
\author{Ganyu (Bruce) Xu (g66xu)}
\date{Fall 2023}

\begin{document}
% Title is not required when submitting to Crowdmark
% \maketitle

\section*{Q5}
\subsection*{(1)}
Denote the columns of $A$ by $A = [\mathbf{a}_1, \mathbf{a}_2, \ldots, \mathbf{a}_n]$. Without loss of generality, let $\mathbf{a}_n$ be a non-zero linear combination of the other $n-1$ columns: $\mathbf{a}_n = A^\prime \mathbf{z}^\prime$ for some $\mathbf{z} \in \mathbb{Z}^{n-1}$.

It is easy to see that because $A^\prime$ contains only a subset of columns of $A$, so $A^\prime\mathbb{Z}^{n-1} \subseteq A\mathbb{Z}^n$. It naturally follows that

$$
A^\prime\mathbb{Z}^{n-1} + q\mathbb{Z}^m \subseteq A\mathbb{Z}^n + q\mathbb{Z}^m
$$


On the other hand, let $\mathbf{v} \in A \mathbb{Z}^{n} + q\mathbb{Z}^m$, then there exist $\mathbf{x}_1 \in \mathbb{Z}^n, \mathbf{x}_2 \in \mathbb{Z}^m$ such that 

$$
\begin{aligned}
\mathbf{v} &= A\mathbf{x}_1 + q\mathbf{x}_2 \\
&= \sum_{i=1}^n(\mathbf{a}_i \mathbf{x}_{(1, i)}) + q\mathbf{x}_2 \\
&= (\sum_{i=1}^{n-1}\mathbf{a}_i x_{(1, i)}) + \mathbf{a}_n x_{(1, n)} + q\mathbf{x}_2 \\
&= A^\prime \cdot (x_{(1, 1)}, x_{(1, 2)}, \ldots, x_{(1, n-1)}) + A^\prime\mathbf{z}^\prime x_{(1, n)} + q\mathbf{x}_2 \\
&= A^\prime((x_{(1, 1)}, x_{(1, 2)}, \ldots, x_{(1, n-1)}) + \mathbf{z}^\prime x_{(1, n)}) + q\mathbf{x}_2 \in A^\prime\mathbb{Z}^{n-1} + q\mathbb{Z}^m
\end{aligned}
$$

Therefore we have $A\mathbb{Z}^{n} + q\mathbb{Z}^m \subseteq A^\prime\mathbb{Z}^{n-1} + q\mathbb{Z}^m$, and the two lattices are indeed equal.

\subsection*{(2)}
For the remainder of this problem, we assume that full-rank LWE with parameters $(m, n, q, U_s, \chi_e)$ exist, which means that $n \leq m$.

Let $(A, \mathbf{b})$ be a sample from generic (aka potentially not full-rank) $\text{LWE}(m, n, q, U_s, \chi_e)$. Without loss of generality, assume that $A = [A_1 \mid A_2] \in \mathbb{Z}_q^{m \times (n_1 + n_2)}$ where $A_1$ is full-rank, and $A_2 = A_1B$ for some non-zero $B \in \mathbb{Z}_q^{n_1 \times n_2}$. Denote the secret by $\mathbf{s} = [\mathbf{s}_1 \mid \mathbf{s}_2]$ where $\mathbf{s}_1 \leftarrow \chi_s^{n_1}, \mathbf{s}_2 \leftarrow \chi_s^{n_2}$, then:

$$
\begin{aligned}
\mathbf{b} &= A\mathbf{s} + \mathbf{e} \\
&= (A_1\mathbf{s}_1 + A_2\mathbf{s}_2) + \mathbf{e} \\
&= A_1\mathbf{s}_1 + A_1B\mathbf{s}_2 + \mathbf{e} \\
&= A_1(\mathbf{s}_1 + B\mathbf{s}_2) + \mathbf{e}
\end{aligned}
$$

we can discard the linearly dependent columns of $A$ and feed the truncated sample $(A_1, \mathbf{b})$ into the full-rank LWE oracle. If the corresponding full-rank Search-LWE has unique solution denoted by $\mathbf{s}^\prime$, then it must be that $\mathbf{b} - A_1\mathbf{s}^\prime = \mathbf{e}$, where $\mathbf{e}$ is exactly the error term from the original Search-LWE instance $(A, \mathbf{b})$.

With the error term recovered, the Search-LWE instance becomes solving noiseless linear equations. Because $A$ is not necessarily full-rank, there may be more than one solutions, but since the secret is uniformly randomly sampled, each solution is equally likely the true secret. I claim that this is the best we can do in terms of solving Search-LWE for non-full-rank $A$.

\end{document}