\documentclass{article}
\usepackage[margin=1in,letterpaper]{geometry}
\usepackage{amsmath,amsfonts,amssymb,amsthm}


\newcommand{\round}[1]{\lfloor {#1} \rceil}
\newcommand{\norm}[1]{\Vert {#1} \Vert}
\newcommand{\var}[1]{\operatorname{Var}[{#1}]}

% For code
\usepackage{listings}

\title{CO 789, Homework 1}
\author{Ganyu (Bruce) Xu (g66xu)}
\date{Fall 2023}

\begin{document}
% Title is not required when submitting to Crowdmark
% \maketitle

\section*{Q8}
Recall that the decryption is computed with the following routine:

$$
\begin{aligned}
D(\text{sk}, (\mathbf{c}_1, c_2))
&= c_2 - \mathbf{c}_1 \cdot \mathbf{s} \\
&= ({\mathbf{s}^\prime}^\intercal\mathbf{e} - {\mathbf{e}^\prime}^\intercal\mathbf{s}) + e^{\prime\prime} + m\round{\frac{q}{2}}
\end{aligned}
$$

Where $\mathbf{s}, \mathbf{s}^\prime$ are coordinate-wise drawn from the secret distribution and $\mathbf{e}, \mathbf{e}^\prime, e^{\prime\prime}$ are coordinate-wise drawn from the error distribution.

According to the decryption routine, a decryption error occurs if and only if the "noise" $({\mathbf{s}^\prime}^\intercal\mathbf{e} - {\mathbf{e}^\prime}^\intercal\mathbf{s}) + e^{\prime\prime}$ exceeds $\round{\frac{q}{4}}$. So to find a modulus that guarantees correct decryption all the time, we need to find the upper bound of noise.

With (baby) Kyber-512, $\chi_s = \mathcal{B}(n=6, p=0.5), \chi_e = \mathcal{B}(n=4, p=0.5)$. Noise is maximized when all entries of $\mathbf{s}, \mathbf{e}$ reach the extremes of the support of their respective distributions. For example, with $\mathbf{s} = \mathbf{s}^\prime = (3, 3, \ldots, 3)$, $\mathbf{e} = (2, 2, \ldots, 2)$, $\mathbf{e}^\prime = (-2, -2, \ldots, -2)$, $e^{\prime\prime} = 2$, the noise term evaluates to $(6 \cdot 512 + 6 \cdot 512) + 2 = 6146$. The smallest prime $q$ such that $\round{\frac{q}{4}} \geq 6146$ is $24593$.


\end{document}