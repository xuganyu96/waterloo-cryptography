\documentclass{article}
\usepackage[margin=1in,letterpaper]{geometry}
\usepackage{amsmath,amsfonts,amssymb,amsthm}

% For code
\usepackage{listings}

\title{CO 789, Homework 1}
\author{Ganyu (Bruce) Xu (g66xu)}
\date{Fall 2023}

\begin{document}
% Title is not required when submitting to Crowdmark
% \maketitle

\section*{Q3}
Recall that a centered binomial distribution is a binomial distribution left-shifted by its mean. Let $X \leftarrow \mathcal{B}(n, p)$ be some binomial distribution, then the centered binomial distribution is described by $Y = X - E[X]$:

$$
P[Y=y] = P[X = y + \mu] = C(n, y + \mu)p^{y + \mu}(1-p)^{n - y - \mu}
$$

\subsection*{(1)}
The probability mass function (PMF) of a centered binomial distribution $X \leftarrow \mathcal{B}(n=6, p=0.5)$ is given by:

$$
P(X=x) = \binom{n}{x + np}p^{x + np}(1-p)^{n - x - np} = \binom{6}{x + 3}2^{-6}
$$

On the other hand, the PMF of a discrete Gaussian with $N(\mu=3, \sigma^2=\frac{3}{2})$ is given by:

$$
P(X=x) = \frac{\rho(x)}{\sum_{j=0}^{q-1}\rho(y)}
$$

I used some Python code to approximate the statistical distance:

\lstinputlisting[language=Python]{q3_1.py}

The result is {0.017725703977230414}.

\subsection*{(2)}
I claim without proof that the most likely error $\mathbf{s} \leftarrow \chi_e^m$, is obtained by sampling the most likely value for each of the entry in. Assuming individual entries of $\mathbf{s}$ are independently sampled from identical distribution $\chi_e$ (a centered binomial distribution), the most likely value for a single entry is $0$. Therefore, the most likely secret is $\mathbf{s} = \mathbf{0} \in \mathbb{F}_q^n$.
https://www.overleaf.com/project/65a54df4dabb2a4ef8fbbc0c
The probability of drawing $\mathbf{0} \leftarrow \mathcal{B}(6, \frac{1}{2})$ is the product of drawing 512 $0$'s:

$$
P(\mathbf{s} = \mathbf{0}) = (\frac{5}{16})^{512}
$$

\subsection*{(3)}
Assume that $\mathbf{s} \leftarrow \mathbb{F}_q^n$ where $n = 512$. The probability of drawing a single $0$ from a centered binomial distribution $\mathcal{B}(6, \frac{1}{2})$ is:

$$
P(Y=0) = P(X = 0 + 3) = C(6, 3)(\frac{1}{2})^3(\frac{1}{2})^3 = \frac{5}{16}
$$

Since each entry of $\mathbf{s} \leftarrow \mathbb{F}_q^{512}$ is independently sampled from this centered binomial distribution, the count of $0$'s in $\mathbf{s}$ also follows a binomial distribution $\mathcal{B}(512, \frac{5}{16})$. The most likely number of $0$ in the secret is thus $512 \cdot \frac{5}{16} = 160$.

In similar fashion, it can be computed that the probability of drawing $1$ from the centered binomial distribution is $C(6, 4)(\frac{1}{2})^6 = \frac{15}{64}$, so the most likely number of $\pm 1$ in the secret is $512 \cdot \frac{15}{64} = 120$, of $\pm 2$ is 48, of $\pm 3$ is 8.

\subsection*{(4)}
\subsubsection*{(a)}
A guess $\hat{\mathbf{s}} \leftarrow \mathbb{F}_q^n$ is correct if the corresponding error term $\hat{\mathbf{e}} \leftarrow \mathbf{b} - A\hat{\mathbf{s}}$ is bounded by the centered binomial distribution: $\hat{\mathbf{e}} \in \{-3, -2, \ldots, 2, 3\}^n$.

\subsubsection*{(b)}
The total number of distinct keys with 160 entries being 0, 120 entries being -1, 48 entries being -2, 8 entries being -3, ... is as follows:

$$
n = \frac{512!}{160!120!120!48!48!8!8!}
$$

Assuming the uniqueness of the secret, there is exactly one correct value for for $\mathbf{s}$. The random process of drawing from $n$ distinct keys without replacement, among which exactly 1 key is considered "success", is modeled by the negative hypergeometric distribution with $N = n, K = 1, r = (N - K) = n - 1$. The expectation of such a distribution is $\frac{n-1}{n}$


\end{document}