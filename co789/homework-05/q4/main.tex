\documentclass{article}
\usepackage[margin=1in,letterpaper]{geometry}
\usepackage{amsmath,amsfonts,amssymb,amsthm}

% For source code
\usepackage{listings}

% Algorithms and pseudocode
\usepackage[ruled,vlined]{algorithm2e}
\usepackage{algpseudocode}

% Common shortcuts
\newcommand{\round}[1]{\lfloor {#1} \rceil}
\newcommand{\norm}[1]{\Vert {#1} \Vert}
\newcommand{\var}[1]{\operatorname{Var}[{#1}]}

% Environments: definitions, theorems, propositions, corollaries, lemmas
%    Theorems, propositions, and definitions are numbered within the section
%    Corollaries are numbered within the theorem, though they are rarely used
\newtheorem{definition}{Definition}[section]
\newtheorem{theorem}{Theorem}[section]
\newtheorem*{remark}{Remark}
\newtheorem{corollary}{Corollary}[theorem]
\newtheorem{proposition}{Proposition}[section]
\newtheorem{lemma}{Lemma}[theorem]


\title{CO 789, Homework 1}
\author{Ganyu (Bruce) Xu (g66xu)}
\date{Fall 2023}

\begin{document}
%%%% TITLE %%%%%
\maketitle

\section{First section}
\begin{definition}[Lattice]\label{lattice-def}
A lattice is a discrete subgroup of $\mathbb{R}^n$
\end{definition}

\begin{theorem}[Minkowski's bound]
    let $\mathcal{L}(B)$ be a full-rank lattice with basis $B \in \mathbb{R}^{n \times n}$, and $B^\ast = [\mathbf{b}_1^\ast, \mathbf{b}_2^\ast, \ldots, \mathbf{b}_n^\ast]$ be the Gram-Schmidt orthogonalization of $B$, then
    \begin{equation}
        \lambda_1(\mathcal{L}(B)) \geq \min_{1 \leq i \leq n}\norm{\mathbf{b}_i^\ast}
    \end{equation}
\end{theorem}

\begin{algorithm}
    \SetAlgoLined
    \caption{Euclid's algorithm}
    \KwData{Two positive integers $a$ and $b$}
    \KwResult{The greatest common divisor of $a$ and $b$}
    \While{$b \neq 0$}{
    $r \leftarrow a \mod b$\;
    $a \leftarrow b$\;
    $b \leftarrow r$\;
    }
    \Return $a$\;
\end{algorithm}

\end{document}