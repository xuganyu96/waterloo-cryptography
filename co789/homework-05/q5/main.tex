\documentclass{article}
\usepackage[margin=1in,letterpaper]{geometry}
\usepackage{amsmath,amsfonts,amssymb,amsthm}

% For source code
\usepackage{listings}

% Algorithms and pseudocode
\usepackage[linesnumbered,ruled,vlined]{algorithm2e}
\usepackage{algpseudocode}

% Common shortcuts
\newcommand{\round}[1]{\lfloor {#1} \rceil}
\newcommand{\Norm}[1]{\Vert {#1} \Vert}
\newcommand{\norm}[1]{\vert {#1} \vert}
\newcommand{\var}[1]{\operatorname{Var}[{#1}]}

% Environments: definitions, theorems, propositions, corollaries, lemmas
%    Theorems, propositions, and definitions are numbered within the section
%    Corollaries are numbered within the theorem, though they are rarely used
\newtheorem{definition}{Definition}[section]
\newtheorem{theorem}{Theorem}[section]
\newtheorem*{remark}{Remark}
\newtheorem{corollary}{Corollary}[theorem]
\newtheorem{proposition}{Proposition}[section]
\newtheorem{lemma}{Lemma}[theorem]


\title{CO 789, Homework 1}
\author{Ganyu (Bruce) Xu (g66xu)}
\date{Fall 2023}

\begin{document}
%%%% TITLE %%%%%
% \maketitle

\section*{Question 5}
To clarify the question, we rephrase the erroneous implementation of the McEliece KEM:

\begin{algorithm}
    \caption{Key generation}
    Sample a random Goppa code $\mathcal{C}(g(x), \{\alpha_j\}_{j=1}^n)$ with parity check matrix $H$\;
    \Return{$\text{pk} = H, \text{sk} = \mathcal{C}$}
\end{algorithm}

\begin{algorithm}
    \caption{Encapsulation}
    Sample a random $e$ such that $\norm{e} = t$\;
    $c \leftarrow He$\;
    $K \leftarrow \operatorname{KDF}(m, c, 0)$
    \Return{$(c, K)$}\;
\end{algorithm}

Recall that the parity check matrix is defined by its individual entries:

\begin{equation*}
    H_{i,j} = \frac{\alpha_j^i}{g(\alpha_j)}
\end{equation*}

We can first recover individual values of $\alpha_j$ using adjacent values of the same column. For each of $1 \leq j \leq n$:

\begin{equation*}
    \frac{H_{1,j}}{H_{0,j}} 
    = \frac{\alpha_j}{g(\alpha_j)} \frac{g(\alpha_j)}{1}
    = \alpha_j
\end{equation*}

In addition, we can recover $g(\alpha_j)$ for $1 \leq j \leq n$ by inverting $H_{0,j}$:

\begin{equation*}
    H_{0,j}^{-1} = g(\alpha_j)
\end{equation*}

From the construction of the Goppa code we know that the degree $t$ of the polynomial $g(x)$ is less than the number of $\alpha_j$'s, which means that with $n$ points on this polynomial we can uniquely determined the polynomial $g(x)$. Thus, we have fully recovered all parameters of the Goppa code from its parity check matrix.

Having recovered the Goppa code $\mathcal{C}$ is equivalent to posessing the secret key since the random permutation and row reduction are both omitted (which is equivalent to having $S = I_{n-k}$ and $P = I_n$). In other words, we have recovered the secret key from the public key, which trivially breaks the KEM.
\end{document}