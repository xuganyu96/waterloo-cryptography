\documentclass{article}
\usepackage[margin=1in,letterpaper]{geometry}
\usepackage{amsmath,amsfonts,amssymb,amsthm,mathtools}

% For source code
\usepackage{listings}

% Algorithms and pseudocode
% \usepackage[linesnumbered,ruled,vlined]{algorithm2e}
\usepackage{algorithm}
\usepackage{algpseudocode}
\usepackage{hyperref}

% Custom commands
\usepackage{crypto-primitives}
\usepackage{kyber-algos}

% Environments: definitions, theorems, propositions, corollaries, lemmas
%    Theorems, propositions, and definitions are numbered within the section
%    Corollaries are numbered within the theorem, though they are rarely used
\newtheorem{definition}{Definition}[section]
\newtheorem{theorem}{Theorem}[section]
\newtheorem*{remark}{Remark}
\newtheorem{corollary}{Corollary}[theorem]
\newtheorem{proposition}{Proposition}[section]
\newtheorem{lemma}{Lemma}[theorem]


\title{
    ElGamal cryptosystem
}
\author{
    Ganyu (Bruce) Xu
}
% Leave the date field empty to display the date of compilation
% \date{}

\begin{document}
%%%% TITLE %%%%%
\maketitle

\section{The ElGamal cryptosystem}
The ElGamal cryptosystem is a public key encryption scheme that mainly operates on the discrete log problem. Each instance of the encryption scheme is parameterized by a cyclic group $G$ with prime order $q$, a generator $g$ of this cyclic group. The routines of the encryption scheme is shown in figure \ref{fig:cpa-elgamal-routines}

\begin{figure}[H]
    \begin{algorithm}[H]
        \caption{$\keygen$}
        \begin{algorithmic}[1]
            \State $x \leftsample \mathbb{Z}_q$
            \State $u \leftarrow g^x$
            \State $\pk \leftarrow u, \sk \leftarrow x$
            \State \Return $(\pk, \sk)$
        \end{algorithmic}
    \end{algorithm}
    \begin{algorithm}[H]
        \caption{$\encrypt(\pk = u, m \in G)$}
        \begin{algorithmic}[1]
            \State $y \leftsample \mathbb{Z}_q$
            \State $v \leftarrow g^y$
            \State $w \leftarrow u^y$
                \Comment{$w = g^{xy}$}
            \State $c \leftarrow (v, m \cdot w)$
            \State \Return $c$
        \end{algorithmic}
    \end{algorithm}
    \begin{algorithm}[H]
        \caption{$\decrypt(\sk = x, c)$}
        \begin{algorithmic}[1]
            \State $(c_1, c_2) \leftarrow c$
            \State $\hat{w} \leftarrow c_1^x$
            \State $\hat{m} \leftarrow c_2 \cdot \hat{w}^{-1}$
            \State \Return $\hat{m}$
        \end{algorithmic}
    \end{algorithm}
    \caption{ElGamal encryption scheme is IND-CPA secure if DDH holds}\label{fig:cpa-elgamal-routines}
\end{figure}

The IND-CPA security of the ElGamal cryptosystem depends on the hardness of the following two problems:

\begin{definition}[Computational Diffie-Hellman Problem]
    Let $G$ be a cyclic group with prime order $q$ and generator $g$. Let $x, y \leftsample \mathbb{Z}_q$ be uniformly random samples. Given $g, g^x,g^y$, compute $g^{xy}$
\end{definition}

\begin{definition}[Decisional Diffie-Hellman Problem]
    Let $G$ be a cyclic group with prime order $q$ and generator $g$. Let $x, y, z \leftsample \mathbb{Z}_q$ be uniformly random samples. Given $g, g^x, g^y$, distinguish $g^{xy}$ from $g^z$
\end{definition}

\begin{theorem}
    For every IND-CPA adversary $A$ against the ElGamal cryptosystem, there exists an adversary $B$ against the DDH game such that

    \begin{equation*}
        \texttt{Adv}(A) = 2 \cdot \texttt{Adv}(B)
    \end{equation*}
\end{theorem}

% TODO: maybe write the proof?

\section{CCA-secure ElGamal construction}
The ElGamal cryptosystem presented in figure \ref{fig:cpa-elgamal-routines} is not secure against chosen-ciphertext attacks. For a simple attack, consider an adversary who just obtained the challenge encryption $c = (g^y, m \cdot (g^x)^y)$, where $y \leftsample \mathbb{Z}_q^\ast, m \leftsample G$ are sampled by the challenger, and $g^x$ is the public key. The adversary can pick some non-zero $y^\prime \in \mathbb{Z}_q^\ast$ and compute a distinct encryption of $m$ using the challenge ciphertext:

\begin{equation*}
    c^\prime = (g^y \cdot g^{y^\prime}, m \cdot (g^x)^y \cdot (g^x)^{y^\prime})
    = (g^{y + y^\prime}, m \cdot (g^x)^{y + y^\prime})
\end{equation*}

The adversary can then query the decryption oracle on $c^\prime$, and the decryption oracle will return $m$, which can then be returned to win the OW-CCA game.

\cite{boneh2020graduate} presented a hybrid encryption scheme that combined the ElGamal cryptosystem with an IND-CPA symmetric cipher $\mathcal{E} = (\encrypt, \decrypt)$ into a public-key encryption scheme. This CCA-secure ElGamal cryptosystem (which we will denote by HPKE for short) is parameterized by: \begin{enumerate}
    \item A cyclic group $G$ of prime order $q$ with generator $g$
    \item A symmetric cipher $\mathcal{E} = (\encrypt_s, \decrypt_s)$ defined over $(\mathcal{K}, \mathcal{M}, \mathcal{C})$
    \item A hash function $H: G \rightarrow \mathcal{K}$
\end{enumerate}

The routines are listed in figure \ref{fig:cca-elgamal-routines}

\begin{figure}[H]
    \begin{algorithm}[H]
        \caption{$\keygen$}
        \begin{algorithmic}[1]
            \State $x \leftarrow \mathbb{Z}_q$
            \State $u \leftarrow g^x$
            \State $\pk \leftarrow u$
            \State $\sk \leftarrow x$
            \State \Return $(\pk, \sk)$
        \end{algorithmic}
    \end{algorithm}
    \begin{algorithm}[H]
        \caption{$\encrypt(\pk = u, m \in \mathcal{M})$}
        \begin{algorithmic}[1]
            \State $y \leftsample \mathbb{Z}_q$
            \State $v \leftarrow g^y$
            \State $w \leftarrow u^y$
                \Comment{$w = g^{xy}$}
            \State $k \leftarrow H(w)$
            \State $c^\prime \leftarrow \encrypt_S(k, m)$
            \State $c \leftarrow (v, c^\prime)$
            \State \Return $c$
        \end{algorithmic}
    \end{algorithm}
    \begin{algorithm}[H]
        \caption{$\decrypt(\sk = x, c)$}
        \begin{algorithmic}[1]
            \State $(v, c^\prime) \leftarrow c$
            \State $\hat{w} \leftarrow v^x$
            \State $\hat{k} \leftarrow H(\hat{w})$
            \State $\hat{m} \leftarrow \decrypt_S(\hat{k}, c^\prime)$
            \State \Return $\hat{m}$
        \end{algorithmic}
    \end{algorithm}
    \caption{ElGamal HPKE}\label{fig:cca-elgamal-routines}
\end{figure}

In this construction \emph{the decryption oracle can be used to construct a decisional Diffie-Hellman problem oracle}: \begin{enumerate}
    \item The DDH adversary $A$ receives $g^x, g^y, w$ and needs to decide whether $w$ is $g^{xy}$ or $g^z$
    \item $A$ samples a random message $m \leftarrow \mathcal{M}$ and computes $k \leftarrow H(w)$ and $c \leftarrow \encrypt_s(k, m)$
    \item $A$ queries the decryption oracle on $c$ and receives some ``decryption'' $\hat{m}$
    \item If $\hat{m} = m$, then $w$ is $g^{xy}$, otherwise $w$ is $g^z$. This is because if $w = g^{xy}$, then $k \leftarrow H(w)$ is the correct symmetric key, so the decryption oracle will decrypt correctly. On the other hand, if $w = g^z$, then $k \leftarrow H(w)$ is a uniformly random key, so the decryptino oracle will not decrypt correctly.
\end{enumerate}

Even though the decryption oracle allows an IND-CCA adversary to solve the decisional Diffie-Hellman problem, the hybrid construction remains CCA secure. This is because \emph{to our knowledge today, solving the decisional Diffie-Hellman problem does not give non-negligible advantage to solving the computational Diffie-Hellman problem}. This idea is expressed in a modified assumption:

\begin{definition}[Interactive computational Diffie-Hellman problem]
    Let $G$ be a cyclic group of prime order $q$ with generator $g$. Let $x, y, z \leftsample \mathbb{Z}_q^\ast$ be uniformly random samples. Given $g, g^x, g^y$ and a decisional Diffie-Hellman oracle $\mathcal{O}: (g^x, g^y, w \in \{g^{xy}, g^z\}) \mapsto \llbrack w = g^{xy} \rrbrack$, there is no efficient adversary who can compute $g^{xy}$ with non-negligible advantage.
\end{definition}

Finally we will put everything together into the security theorem for CCA ElGamal.

\begin{theorem}
    Under the random oracle model, for every IND-CCA adversary $A$ against the HPKE, there exists an \emph{interactive computational Diffie-Hellman problem} adversary $B$ and an IND-CPA adversary $C$ against the symmetric encryption scheme such that

    \begin{equation*}
        \texttt{Adv}_\texttt{IND-CCA}(A) \leq \texttt{Adv}_\texttt{ICDH}(B) + \texttt{Adv}_\texttt{IND-CPA}(C)
    \end{equation*}
\end{theorem}

\begin{proof}
    We will prove using a sequence of games. The games are listed in figure \ref{fig:cca-elgamal-sequence}

    \begin{figure}[H]
        \begin{algorithm}[H]
            \caption{Games $G_0 - G_1$}
            \begin{algorithmic}[1]
                \State $x \leftsample \mathbb{Z}_q^\ast$
                    \Comment{$x$ is the secret key}
                \State $u \leftarrow g^x$
                    \Comment{$u$ is the public key}
                \State $(m_0, m_1) \leftarrow A^{\mathcal{O}^\decrypt}(u)$
                \State $y \leftsample \mathbb{Z}_q^\ast$
                \State $v \leftarrow g^y$
                \State $w \leftarrow u^y$
                \State $k \leftarrow H(w)$
                    \Comment{Game 0}
                \State $k \leftsample \mathcal{K}$
                    \Comment{Game 1}
                \State $b \leftsample \{0,1\}$
                \State $c^\prime \leftarrow \encrypt_s(k, m_b)$
                \State $c^\ast \leftarrow (v, c^\prime)$
                \State $\hat{b} \leftarrow A^{\mathcal{O}^\decrypt}(u, c^\ast, (m_0, m_1))$
                \State \Return $\llbrack \hat{b} = b \rrbrack$
            \end{algorithmic}
        \end{algorithm}
        \caption{Sequence of games}\label{fig:cca-elgamal-sequence}
    \end{figure}

    \emph{Game 0} is the standard IND-CCA game: $\texttt{Adv}_0(A) \coloneq \texttt{Adv}_\texttt{IND-CCA}(A)$

    \emph{Game 1} is identical to game 0, except that in the challenge encryption, the symmetric key $k \leftsample \mathcal{K}$ is uniformly random instead of pseudorandomly derived. Under the random oracle model, the two games are statistically indistinguishable from adversary $A$'s perspective unless $A$ queries $H$ on $w = g^{xy}$. Denote this event by $\texttt{QUERY}^\ast$, then by the difference lemma:

    \begin{equation*}
        \texttt{Adv}_0(A) - \texttt{Adv}_1(A) \leq P\left\lbrack \texttt{QUERY}^\ast \right\rbrack
    \end{equation*}

    \emph{Game 1} can be entirely simulated by an IND-CPA adversary $C$ against the symmetric cipher. $C$ can generate the ElGamal keypair on its own, simulate the hash oracle $H$, and service $A$'s decryption queries (using the generated keypair) before $A$ outputs the chosen plaintexts $m_0, m_1$. When $A$ outputs the chosen plaintexts $m_0, m_1$, $C$ outputs them as its own chosen plaintexts and receives the challenge ciphertext $c^\prime$. $C$ then samples random $y \leftsample \mathbb{Z}_q^\ast$, computes $v \leftarrow g^y$, and returns $c^\ast = (v, c^\prime)$ to $A$ as $A$'s challenge encryption. Finally, when $A$ outputs its guess $\hat{b}$, $C$ passes $\hat{b}$ as its own guess. It is easy to see that $C$ wins the IND-CPA game if and only if $A$ wins game 1:

    \begin{equation*}
        \texttt{Adv}_1(A) = \texttt{Adv}_\texttt{IND-CPA}(C)
    \end{equation*}

    We now bound the probability $P\left\lbrack \texttt{QUERY}^\ast \right\rbrack$ by constructing an interactive computational Diffie-Hellman problem adversary $B$ using $A$ as a subroutine. To do that, $B$ needs to service $A$'s hash queries and decryption queries. The simulated hash oracles and decryption oracles are listed in figure \ref{fig:sim-oracles}. Note that $\mathcal{L}^H$ is used to record the queries made to $H$, while $\mathcal{L}^\decrypt$ is used to record symmetric keys used for decryption queries.
    
    \begin{figure}[H]
        \begin{minipage}{0.5\textwidth}
            \begin{algorithm}[H]
                \caption{$\mathcal{O}^H_1(w)$}
                \begin{algorithmic}[1]
                    \If{$\exists (\tilde{w}, \tilde{k}) \in \mathcal{L}^H:\tilde{w} = w$}
                        \State \Return $\tilde{k}$
                    \ElsIf{$\exists (\tilde{v}, \tilde{k}) \in \mathcal{L}^\decrypt
                        :\mathcal{O}^\texttt{DDH}(g^x, \tilde{v}, w) = 1$}
                        \State $k \leftarrow \tilde{k}$
                    \Else
                        \State $k \leftsample \mathcal{K}$
                    \EndIf
                    \State $\mathcal{L}^H \leftarrow \mathcal{L}^H \cup \{(w, k)\}$
                    \State \Return $k$
                \end{algorithmic}
            \end{algorithm}
        \end{minipage}
        \begin{minipage}{0.49\textwidth}
            \begin{algorithm}[H]
                \caption{$\mathcal{O}^\decrypt_1(v, c^\prime)$}
                \begin{algorithmic}[1]
                    \If{$\exists (\tilde{w}, \tilde{k}) \in \mathcal{L}^H
                        :\mathcal{O}^\texttt{DDH}(g^x, v, \tilde{w}) = 1$}
                        \State \Return $\decrypt_s(\tilde{k}, c^\prime)$
                    \ElsIf{$\exists (\tilde{v}, \tilde{k}) \in \mathcal{L}^\decrypt
                        :\tilde{v} = v$}
                        \State \Return $\decrypt_s(\tilde{k}, c^\prime)$
                    \Else
                        \State $k \leftsample \mathcal{K}$
                        \State $\mathcal{L}^\decrypt \leftarrow \mathcal{L}^\decrypt \cup \{(v, k)\}$
                        \State \Return $\decrypt_s(k, c^\prime)$
                    \EndIf
                \end{algorithmic}
            \end{algorithm}
        \end{minipage}
        \caption{Simulated hash oracle and decryption oracle}\label{fig:sim-oracles}
    \end{figure}

    Assuming that $\mathcal{O}^\texttt{DDH}$ is always correct, the IND-CCA adversary $A$ cannot distinguish the simulated oracles from the true oracles.

    When $A$ produces the chosen plaintexts $m_0, m_1$, $B$ needs to perform the challenge encryption: \begin{enumerate}
        \item $v \leftarrow g^y$, where $g^y$ is given to $B$ as part of ICDH input
        \item $k \leftsample \mathcal{K}$ as per game 1
        \item $b \leftsample \{0,1\}; c^\prime \leftarrow \encrypt_s(k, m_b)$
        \item Return $(v, c^\prime)$ as the challenge ciphertext
    \end{enumerate}

    Afterwards, $B$ continues simulating the oracles for $A$ until $A$ halts. Then, $B$ searches through $\mathcal{L}^H$. If $\texttt{QUERY}^\ast$ happens, then there exists $\tilde{w} \in \mathcal{L}^H$ such that $\mathcal{O}^\texttt{DDH}(g^x, g^y, \tilde{w}) = 1$, and $B$ can return $\tilde{w}$ and win the ICDH game. Therefore:

    \begin{equation*}
        P\left\lbrack \texttt{QUERY}^\ast \right\rbrack \leq \texttt{Adv}_\texttt{ICDH}(B)
    \end{equation*}

    Finally, putting the equations above give us the desired result.
\end{proof}

\section{Does this apply to Kyber-AE?}
\emph{Unfortuantely not}.

In the case of the hybrid ElGamal presented in figure \ref{fig:cca-elgamal-routines}, the decryption oracle is converted into a decisional Diffie-Hellman oracle, but it can be safely assumed that having a decisional Diffie-Hellman oracle does not provide non-negligible help. However, there is no comparable "loosening of security assumption" for Kyber-AE, where the decapsulation oracle can be converted into a plaintext-checking oracle against the underlying PKE, which then recovers the secret key.

\bibliographystyle{alpha}
\bibliography{references}

\end{document}