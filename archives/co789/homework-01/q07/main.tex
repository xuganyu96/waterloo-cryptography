\documentclass{article}
\usepackage[margin=1in,letterpaper]{geometry}
\usepackage{amsmath,amsfonts,amssymb,amsthm}

% For code
\usepackage{listings}


\newcommand{\round}[1]{\lfloor {#1} \rceil}
\newcommand{\norm}[1]{\Vert {#1} \Vert}
\newcommand{\var}[1]{\operatorname{Var}[{#1}]}

\title{CO 789, Homework 1}
\author{Ganyu (Bruce) Xu (g66xu)}
\date{Fall 2023}

\begin{document}
% Title is not required when submitting to Crowdmark
% \maketitle

\section*{Q7}
We can set one of the base vectors to be extremely short and all other base vectors extremely long. For ease of constructions we will build orthogonal basis.

Denote the columns of the identity matrix $I_n \in \mathbb{R}^{n \times n}$ by $I = [\mathbf{e}_1, \mathbf{e}_2, \ldots, \mathbf{e}_n]$.

Let $\mathcal{L}(\{\mathbf{b}_1, \mathbf{b}_2, \ldots, \mathbf{b}_n\}) \subset \mathbb{R}^n$ be spanned by the basis $\mathbf{b}_1, \mathbf{b}_2, \ldots, \mathbf{b}_n$, where $\mathbf{b}_1 = \mathbf{e}_1$, and $\mathbf{b}_i = 3R\mathbf{e}_i$ for $i > 1$. Finally, set $\mathbf{v} = (0, 1.1R, 0, 0, \ldots, 0)$.

It is easy to see that $\mathbf{b}_1, \mathbf{b}_2, \ldots, \mathbf{b}_n$ form an orthogonal basis. The shortest vector in $\mathcal{L}$ is $\mathbf{b}_1$, so $\lambda_1(\mathcal{L}) = 1$. The closest lattice point to $\mathbf{v}$ is $\mathbf{0}$, so the distance between $\mathbf{v}$ and $\mathcal{L}$ is $\norm{\mathcal{L} - \mathbf{v}} = 1.1R > R$


\end{document}