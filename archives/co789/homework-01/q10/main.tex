\documentclass{article}
\usepackage[margin=1in,letterpaper]{geometry}
\usepackage{amsmath,amsfonts,amssymb,amsthm}

% For code
\usepackage{listings}


\newcommand{\round}[1]{\lfloor {#1} \rceil}
\newcommand{\norm}[1]{\Vert {#1} \Vert}
\newcommand{\var}[1]{\operatorname{Var}[{#1}]}

\title{CO 789, Homework 1}
\author{Ganyu (Bruce) Xu (g66xu)}
\date{Fall 2023}

\begin{document}
% Title is not required when submitting to Crowdmark
% \maketitle

\section*{Q10}
Recall from textbook LWE the decryption routine:

$$
\begin{aligned}
D(\text{sk}, (\mathbf{c}_1, c_2))
&= c_2 - \mathbf{c}_1\mathbf{s} \\
&= (\mathbf{s}_1^\intercal\mathbf{b} + e^\prime + m\round{\frac{q}{2}}) 
    - (\mathbf{s}_1^\intercal A\mathbf{s} + \mathbf{e}_1^\intercal\mathbf{s}) \\
&= (\mathbf{s}_1^\intercal(A\mathbf{s} + \mathbf{e}) + e^\prime + m\round{\frac{q}{2}}) 
    - (\mathbf{s}_1^\intercal A\mathbf{s} + \mathbf{e}_1^\intercal\mathbf{s}) \\
&= \mathbf{s}_1^\intercal\mathbf{e} + e^\prime + m\round{\frac{q}{2}} - \mathbf{e}_1^\intercal\mathbf{s} \\
&= (\mathbf{s}_1^\intercal\mathbf{e} - \mathbf{e}_1^\intercal\mathbf{s}) + e^\prime + m\round{\frac{q}{2}}
\end{aligned}
$$

Also recall the definition of the rounding operator: $\round{a} = a + \delta$ for some $-\frac{1}{2} \leq \delta < \frac{1}{2}$.

\subsection*{(1)}
Where $\mathbf{e}_1$ is maliciously set to all zeros except for the i-th component, $\mathbf{e}_1^\intercal \mathbf{s}$ evaluates to $s_i \cdot \round{\frac{q}{2}}$, where $s_i$ is the i-th component of the LWE secret $\mathbf{s}$.

If $s_i = 2k$ is even, then

$$
\begin{aligned}
D(\text{sk}, (\mathbf{c}_1, c_2)) 
&= (\mathbf{s}_1^\intercal\mathbf{e} - \mathbf{e}_1^\intercal\mathbf{s}) + e^\prime + m\round{\frac{q}{2}} \\
&= \mathbf{s}_1^\intercal\mathbf{e} + e^\prime - 2k(\frac{q}{2} + \delta) + m\round{\frac{q}{2}} \\
&\equiv  \mathbf{s}_1^\intercal\mathbf{e} + e^\prime - 2k\delta + m\round{\frac{q}{2}} \\
\end{aligned}
$$

Thus, where $k$ is sufficiently small (corresponding to $s_i$ being small), $\mathbf{s}_1^\intercal\mathbf{e} + e^\prime + 2k\delta$ has a high probability of being less than $\frac{q}{4}$, so the ciphertext will be correctly decrypted.

If $s_i = 2k + 1$ is odd, then

$$
\begin{aligned}
D(\text{sk}, (\mathbf{c}_1, c_2)) 
&= (\mathbf{s}_1^\intercal\mathbf{e} - \mathbf{e}_1^\intercal\mathbf{s}) + e^\prime + m\round{\frac{q}{2}} \\
&= \mathbf{s}_1^\intercal\mathbf{e} + e^\prime + (2k + 1)\round{\frac{q}{2}} + m\round{\frac{q}{2}} \\
&\equiv  \mathbf{s}_1^\intercal\mathbf{e} + e^\prime + 2k\delta + \round{\frac{q}{2}} + m\round{\frac{q}{2}} \\
\end{aligned}
$$

Notice the additional $\round{\frac{q}{2}}$ term in the R.H.S., which will cause decryption to be incorrect with high probability. Thus, with high probability, decryption will be correct if and only if $s_i$ is even.

\subsection*{(2)}
Eve can recover the parity of all entries of the secret $\mathbf{s} \in \chi_s^n$ by preparing $n$ emails $\{(\mathbf{c}_{(i, 1)}, c_{(i, 2)})\}_{i=1}^{n}$ where $(\mathbf{c}_{(i, 1)}, c_{(i, 2)})$ is generated using the procedure described in part (1): $\mathbf{e}_{(i, 1)}$ is all 0's except for the i-th entry, which is set to $\round{\frac{q}{2}}$. For each of these $n$ emails, Eve sends it to Alice's server and receives the auto-response. The quoted email in the auto-response is checked against the original email: if they are identical, then Alice's server correctly decrypted Eve's email, so the corresponding entry in $\mathbf{s}$ is even; otherwise, the corresponding entry in $\mathbf{s}$ is odd.

\subsection*{(3)}
We will describe an algorithm for recovering the value of a single component of the secret key $\mathbf{s}$. From here, the algorithm can be repeated $n$ times to recover the value of all components of the secret key. If the algorithm for recovering a single value is efficient, then repeating it $n$ times is also efficient.

Consider the ciphertext $(\mathbf{c}_1, c_2)$ where all entries of $\mathbf{c}_1$ is set to 0 except for the i-th entry, which we denote by $\mathbf{c}_{(1, i)}$. Recall from the decryption procedure the fact that decryption outputs $0$ if and only if $-\round{\frac{q}{4}} \leq m^\prime \leq \round{\frac{q}{4}}$, which is equivalent to:

$$
-\round{\frac{q}{4}} + c_2 \leq c_{(1, i)} \cdot s_i \leq \round{\frac{q}{4}} + c_2
$$

Where $s_i$ is the i-th entry of the secret $\mathbf{s}$.

In other words, we can check whether $s_i$ falls within a certain range of values given some values of $c_2$ and $c_{(1, i)}$. In addition, changing the value of $c_2$ will translate the range, while changing the value of $c_{(1, i)}$ will scale the range. Thus, we can perform a binary search by adjusting the values of $c_2, c_{(1, i)}$ and pinpoint the value of $s_i$ in $O(\log{q})$ steps.

\subsection*{(4)}
Converting textbook LWE to be resistant to chosen-ciphertext attack is non-trivial. In Kyber/Dilithium a tweaked version of Fujisaki-Okamoto transformation is applied to the textbook LWE to obtain a CCA2-secure cryptosystem. Due to time constraint, the details will not be covered in this write-up.


\end{document}