\documentclass{article}
\usepackage[margin=1in,letterpaper]{geometry}
\usepackage{amsmath,amsfonts,amssymb,amsthm}

% For code
\usepackage{listings}

\title{CO 789, Homework 1}
\author{Ganyu (Bruce) Xu (g66xu)}
\date{Fall 2023}

\begin{document}
% Title is not required when submitting to Crowdmark
% \maketitle

\section*{Q1}
\subsection*{(1)}
The probability formula will be proved by induction. The base case is trivial: an empty set is always linearly independent, so the probability of drawing a linearly independent empty set is exactly 1.

For the induction, assume that $\mathbf{a}_1, \mathbf{a}_2, \ldots, \mathbf{a}_{t-1} \in \mathbb{F}_q^n$ are linearly independent, then consider the probability of drawing a t-th vector uniformly from $\mathbb{F}_q^n$ such that it is not in the linear span of the previous $t-1$ vectors: $\mathbf{a}_t \notin \operatorname{Span}(\mathbf{a}_1, \mathbf{a}_2, \ldots, \mathbf{a}_{t-1})$.

There are a total of $q^n$ possible values for $\mathbf{a}_t$ to draw from. On the other hand, there are a total of $q^{t-1}$ possible combinations of coefficients (including all zeros) for $t-1$ vectors. Singe $\mathbf{a}_1, \mathbf{a}_2, \ldots, \mathbf{a}_{t-1}$ are linearly independent, each unique combination of coefficients corresponds to a unique element in the linear span. Thus, there are a total of $q^n - q^{t-1}$ possible values that are outside the linear span of $\mathbf{a}_1, \mathbf{a}_2, \ldots, \mathbf{a}_{t-1}$, and the probability of uniformly drawing a $\mathbf{a}_t$ that's outside the linear span is $1 - \frac{q^{t-1}}{q^n}$.

In other words:

$$
P(
    \mathbf{a}_t 
    \notin \operatorname{Span}(\mathbf{a}_1, \mathbf{a}_2, \ldots, \mathbf{a}_{t-1})
    \mid \{\mathbf{a}_i\}_{i=1}^{t-1} \text{ is linearly independent}
) = 1 - q^{t-1-n}
$$

From here, we can recursively compute the probability of drawing $m$ linearly independent vectors:

$$
\begin{aligned}
&P(\{\mathbf{a}_i\}_{i=1}^m \text{ is linearly independent}) \\
&= \prod_{j=0}^{m-1} P(
    \mathbf{a}_{j+1} 
    \notin \operatorname{Span}(\{\mathbf{a}_i\}_{i=1}^{j})
    \mid \{\mathbf{a}_i\}_{i=1}^{j} \text{ is linearly independent}
) \\
&= \prod_{j=0}^{m-1}(1 - q^{(j+1) - 1 - n}) \\
&= \prod_{j=0}^{m-1}(1 - q^{j - n})
\end{aligned}
$$

\subsection*{(2)}
Notice from the probability formula above:

$$
\begin{aligned}
\prod_{i=0}^{n-1} (1 - q^{i-n})
&= (1-q^{-n}) \cdot \prod_{i=1}^{n-1}(1 - q^{i-n}) \\
&= (1-q^{-n}) \cdot \prod_{i=0}^{n-2}(1 - q^{i - (n - 1)}) \\
\end{aligned}
$$

The product in the R.H.S. is the probability of drawing $n-1$ linearly independent vectors from $\mathbf{Z}_q^{n-1}$. Since $1 - q^{-n} < 1$ for all $q, n > 0$, the value of this probability strictly decreases as $n$ increases. Therefore, we can bound the probability from below by setting $n$ to the maximal value $1024$:

$$
\begin{aligned}
\prod_{i=0}^{n-1} (1 - q^{i-n}) 
&\geq \prod_{i=0}^{1023} (1 - q^{i-1024}) \\
&= (1-q^{-1024})(1-q^{-1023}) \ldots (1-q^{-1})
&\geq (1 - q^{-1})^{1024} \\
&= (1 - 3329^{-1})^{1024} \\
&\approx 0.735175 > \frac{2}{3}
\end{aligned}
$$

\subsection*{(3)}
For Kyber-512, the parameters are defined with $n=256, q=3329$. Plugging them into the formula:

\lstinputlisting[language=Python]{q1_3.py}

The result is $0.999699519257883$

\end{document}