\documentclass{article}
\usepackage[margin=1in,letterpaper]{geometry}
\usepackage{amsmath,amsfonts,amssymb,amsthm}

% For code
\usepackage{listings}

\title{CO 789, Homework 1}
\author{Ganyu (Bruce) Xu (g66xu)}
\date{Fall 2023}

\begin{document}
% Title is not required when submitting to Crowdmark
% \maketitle

\section*{Q6}
\subsection*{(1)}
Let $X \in \mathcal{B}(n, p)$ be a random variable that follows binomial distribution. Recall from the definition of a binomial distribution that $X = I_1 + I_2 + \ldots + I_n$ where each of $I_i$ is an independent coin toss with PMF:

$$
\begin{cases}
P(I_i=1) = p \\
P(I_i=0) = 1-p
\end{cases}
$$

Let $X_1 \in \mathcal{B}(n_1, p), X_2 \in \mathcal{B}(n_2, p)$ be two independent random variables following binomial distributions, then:

$$
\begin{aligned}
X_1 &= \sum_{i=1}^{n_1} I_i \\
X_2 &= \sum_{i=n_1 + 1}^{n_1 + n_2} I_i \\
\end{aligned}
$$

Therefore $X_1 + X_2 = \sum_{i=1}^{n_1 + n_2} I_i$, which is a binomial distribution $\mathcal{B}(n_1 + n_2, p)$.

Recall that centered binomial distribution is defined by subtracting the corresponding binomial distribution by a constant (the expectation of said binomial distribution): $C_i = X_i - E(X_i)$. Therefore, given centered binomial distributions $C_1 = X_1 - E[X_1], C_2 = X_2 - E[X_2]$:

$$
\begin{aligned}
C_1 + C_2 &= X_1 + X_2 - E[X_1] - E[X_2] \\
&= (X_1 + X_2) - E[X_1 + X_2]
\end{aligned}
$$

From the results above we know that because $X_1, X_2$ are independent binomial distributions, $X_1 + X_2$ follows binomial distribution $\mathcal{B}(n_1 + n_2, p)$, thus $C_1 + C_2$ follows centered binomial distribution $\mathcal{B}(n_1 + n_2, p)$.

\subsection*{(2)}
From part (a) we know that the sum of $k$ i.i.d. random variables following binomial $\mathcal{B}(n, p)$ is a random variable following binomial $\mathcal{B}(kn, p)$.

Let $I^m = \{0, 1\}^m$ denote the set of bit-masks with length $m$ and $K \subseteq I$ denote the subset of vectors with exactly $k$ entries being $1$, then $\vert K \vert = \binom{m}{k}$. Thus we iterate through all possible values in $\mathbf{k} \in K$ and compute the inner product $\mathbf{e}^\intercal \mathbf{k}$. Since $\mathbf{k}$ has exactly $k$ entries being 1 and $\mathbf{e}$ contains $m$ independent samples from centered binomial $(n, p)$, $\mathbf{e}^\intercal\mathbf{k}$ is the sum of $k$ i.i.d. centered binomial with parameters $(kn, p)$


\end{document}