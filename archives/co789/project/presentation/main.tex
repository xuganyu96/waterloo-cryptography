\documentclass{beamer}
% \usetheme{Copenhagen}

%Information to be included in the title page:
\title{Security reduction of FO transform and variations}
\author{Ganyu (Bruce) Xu}
\institute{University of Waterloo}
\date{April, 2024}

% For writing algorithm
\usepackage[ruled,vlined]{algorithm2e}
% For using strikethrough
\usepackage[normalem]{ulem}

% My own macros
\newcommand{\leftsample}{\overset{{\scriptscriptstyle\$}}{\leftarrow}}


\begin{document}


\frame{\titlepage}

\begin{frame}
    \frametitle{Fujisaki-Okamoto transformation, 1999}

    Inputs:
    \begin{itemize}
        \item Public-key encryption scheme: $(\operatorname{KeyGen}, E^\text{asym}, D^\text{asym})$
        \item Symmetric cipher $(E^\text{sym}, D^\text{sym})$
        \item Key-derivation function \footnote{This is also a hash function and follows the random oracle assumption} $G: \{0, 1\}^\ast \rightarrow \mathcal{K}^\text{sym}$
        \item Hash function $H: \{0, 1\}^\ast \rightarrow \text{Coin}^\text{asym}$
    \end{itemize}

    $\linebreak$

    Hybrid scheme's key generation is identical to the PKE's    
\end{frame}

\begin{frame}
    \frametitle{FO 1999 routines}

    \begin{columns}
        \column{0.6\textwidth}
        \begin{algorithm}[H]
        \SetAlgoLined
        \caption{$E^\text{hy}$}

        \KwIn{$\text{pk}^\text{hy}, m \in \mathcal{M}^\text{sym}$}
        \KwOut{$(e \in \mathcal{C}^\text{asym}, c \in \mathcal{C}^\text{sym})$}
        $\sigma \leftsample \mathcal{M}^\text{asym}$\;
        $
            a \leftarrow G(\sigma),
            c \leftarrow E^\text{sym}_a(m)
        $\;
        \tcp{PKE encryption accepts $r$ as a seed}
        $
            r \leftarrow H(c, \sigma), 
            e \leftarrow E^\text{asym}(\operatorname{pk}, \sigma, r)
        $
        \;
        \Return{$(e, c)$}\;
        \end{algorithm}

        \column{0.4\textwidth}
        \begin{algorithm}[H]
        \SetAlgoLined
        \caption{$D^\text{hy}$}

        \KwIn{$\text{pk}, \text{sk}, (e, c)$}
        
        $\hat{\sigma} \leftarrow D^\text{asym}(\text{sk}, e)$\;
        $\hat{r} \leftarrow H(c, \hat{\sigma})$\;
        $\hat{c} \leftarrow E^\text{asym}(\text{pk}, \hat{\sigma}, \hat{r})$\;
        \If{$\hat{c} \neq c$}{
            \Return{$\bot$}\;
        }
        $\hat{a} \leftarrow G(\sigma)$\;
        $\hat{m} \leftarrow D^\text{sym}_{\hat{a}}(c)$\;
        \Return{$\hat{m}$}\;
        \end{algorithm}
    \end{columns}
    
    

\end{frame}

\begin{frame}
    \frametitle{Fujisaki-Okamoto transformation, 1999}


    \begin{alertblock}{Security result}
        Under the random oracle assumption, for every IND-CCA adversary against the hybrid scheme with advantage $\epsilon^\text{hy}_\text{IND-CCA}$, there exists an OW-CPA adversary against the underlying PKE with advantage $\epsilon^\text{asym}_\text{OW-CPA}$ and an IND-CPA adversary against the underlying symmetric ciphert with advantage $\epsilon^\text{sym}_\text{IND-CPA}$ such that

        \begin{equation*}
             \epsilon^\text{hy}_\text{IND-CCA} 
             \leq q_D2^{-\gamma} 
                + q_H\epsilon^\text{asym}_\text{OW-CPA} 
                + \epsilon^\text{sym}_\text{IND-CPA} 
        \end{equation*}

        % NOTE: I will need to explain the terms
    \end{alertblock}

\end{frame}

\begin{frame}
    \frametitle{Fujisaki-Okamoto transformation, 1999}

    Proof overview:
    \begin{itemize}
        \item Use $\mathcal{A}^\text{asym}_\text{OW-CPA}$ and $\mathcal{A}^\text{sym}_\text{IND-CPA}$ to simulate the IND-CCA game
        \item Simulate decryption oracle without using secret key
    \end{itemize}
\end{frame}

\begin{frame}
    \frametitle{Fujisaki-Okamoto transformation, 1999}

    To simulate $\mathcal{O}^D(e, c)$ without secret key:
    
    % TODO: find a way to reset counter
    \begin{algorithm}[H]
    \SetAlgoLined
    \caption{Hybrid encryption $E^\text{hy}$}

    \KwIn{$\text{pk}^\text{hy}, m \in \mathcal{M}^\text{sym}$}
    \KwOut{$(e \in \mathcal{C}^\text{asym}, c \in \mathcal{C}^\text{sym})$}
    $\sigma \leftsample \mathcal{M}^\text{asym}$\;
    $
        a \leftarrow G(\sigma),
        c \leftarrow E^\text{sym}_a(m)
    $\;
    \tcp{PKE encryption accepts $r$ as a seed}
    \alert{$r \leftarrow H(c, \sigma)$}
    $
        , 
        e \leftarrow E^\text{asym}(\operatorname{pk}, \sigma, r)
    $
    \;
    \Return{$(e, c)$}\;
    \end{algorithm}
\end{frame}

\begin{frame}
    \frametitle{Decryption oracle without secret key}

    \begin{algorithm}[H]
    \SetAlgoLined
    \caption{$\mathcal{O}^D_1$: decryption oracle without sk}
    \KwIn{The query $(\tilde{e}, \tilde{c})$}
    
    \ForEach{$(\sigma, c, r)$ in $H$'s tape}{
        \If{$c = \tilde{c}$}{
            $a \leftarrow G(\sigma)$\;
            $m \leftarrow D^\text{sym}_a(\tilde{c})$\;
            \Return{$m$}\;
        }
    }
    \Return{$\bot$}\;
    \end{algorithm}
\end{frame}

\begin{frame}
    \frametitle{Challenge encryption with truly random key/coin}

    \begin{algorithm}[H]
    \SetAlgoLined
    \caption{Challenge encryption $E^\text{hy}_\ast$}

    \KwIn{$\text{pk}^\text{hy}, m \in \mathcal{M}^\text{sym}$}
    \KwOut{$(e \in \mathcal{C}^\text{asym}, c \in \mathcal{C}^\text{sym})$}
    $\sigma \leftsample \mathcal{M}^\text{asym}$\;
    $
        \alert{a \leftsample \mathcal{K}^\text{sym}},
        c \leftarrow E^\text{sym}_a(m)
    $\;
    \tcp{PKE encryption accepts $r$ as a seed}
    $
        \alert{r \leftsample \text{Coin}}, 
        e \leftarrow E^\text{asym}(\operatorname{pk}, \sigma, r)
    $
    \;
    \Return{$(e, c)$}\;
    \end{algorithm}

\end{frame}

\begin{frame}
    \frametitle{Game 0: IND-CCA game}

    \begin{algorithm}[H]
    \SetAlgoLined
    \caption{Vanilla IND-CCA game}
    $(\text{pk}, \text{sk}) \leftsample \operatorname{KeyGen}()$\;
    $(m_0, m_1) \leftsample
        \mathcal{A}^\text{hy}_\text{IND-CCA}(\text{pk}, \mathcal{O}^D)
    $\;
    $b \leftsample \{0, 1\}$\;
    $c^\ast \leftarrow E^\text{hy}(\text{pk}, m_b)$\;
    $\hat{b} \leftsample \mathcal{A}^\text{hy}_\text{IND-CCA}(
        \text{pk}, \mathcal{O}^D, c^\ast
    )$\;
    Adversary wins if $\hat{b} = b$\;

    \end{algorithm}

\end{frame}

\begin{frame}
    \frametitle{Game 1: modify the decryption oracle}

    \begin{algorithm}[H]
    \SetAlgoLined
    \caption{Game 1}
    $(\text{pk}, \text{sk}) \leftsample \operatorname{KeyGen}()$\;
    $(m_0, m_1) \leftsample
        \mathcal{A}^\text{hy}_\text{IND-CCA}(\text{pk}, \alert{\mathcal{O}^D_1})
    $\;
    $b \leftsample \{0, 1\}$\;
    $c^\ast \leftarrow E^\text{hy}(\text{pk}, m_b)$\;
    $\hat{b} \leftsample \mathcal{A}^\text{hy}_\text{IND-CCA}(
        \text{pk}, \alert{\mathcal{O}^D_1}, c^\ast
    )$\;
    Adversary wins if $\hat{b} = b$\;

    \end{algorithm}

    Loss of security when $\mathcal{A}$ queries $\mathcal{O}^D$ with valid ciphertexts built without querying $H$ at least once

    \begin{equation*}
        \epsilon_0 - \epsilon_1 \leq q_D2^{-\gamma}
    \end{equation*}
    
    % NOTE: need to explain gamma
\end{frame}

\begin{frame}
    \frametitle{Game 2: use true randomness in challenge encryption}

    \begin{algorithm}[H]
    \SetAlgoLined
    \caption{Game 2}
    $(\text{pk}, \text{sk}) \leftsample \operatorname{KeyGen}()$\;
    $(m_0, m_1) \leftsample
        \mathcal{A}^\text{hy}_\text{IND-CCA}(\text{pk}, \alert{\mathcal{O}^D_1})
    $\;
    $b \leftsample \{0, 1\}$\;
    $c^\ast \leftarrow \alert{E^\text{hy}_\ast}(\text{pk}, m_b)$\;
    $\hat{b} \leftsample \mathcal{A}^\text{hy}_\text{IND-CCA}(
        \text{pk}, \alert{\mathcal{O}^D_1}, c^\ast
    )$\;
    Adversary wins if $\hat{b} = b$\;

    \end{algorithm}
    
    Loss of security when $\mathcal{A}$ queries either $G$ or $H$ with $\sigma^\ast$

    \begin{equation*}
        \epsilon_1 - \epsilon_2 \leq P[\text{QUERY}^\ast]
    \end{equation*}
\end{frame}

\begin{frame}
    \frametitle{Simulate game 2 with IND-CPA adversary}

    \begin{algorithm}[H]
    \SetAlgoLined
    \caption{Symmetric cipher IND-CPA game $(E^\text{sym}, D^\text{sym})$}
    $a^\ast \leftsample \mathcal{K}^\text{sym}$\;
    $(\text{pk}, \text{sk}) \leftsample \operatorname{KeyGen}^\text{hy}()$\;
    $(m_0, m_1) \leftsample \mathcal{A}^\text{hy}_\text{IND-CCA}(
        \text{pk}, \mathcal{O}^D_1
    )$\;
    $b \leftsample \{0, 1\}$\;
    $c^\ast \leftarrow E^\text{sym}_{a^\ast}(m_b)$\;
    $
        \sigma^\ast \leftsample \mathcal{M}^\text{asym},
        r^\ast \leftsample \text{Coin}
    $
    \;
    $e^\ast \leftarrow E^\text{asym}(\text{pk}, \sigma^\ast, r^\ast)$\;
    $
        \hat{b} \leftsample \mathcal{A}^\text{hy}_\text{IND-CCA}(
            \text{pk}, \mathcal{O}^D_1, (e^\ast, c^\ast)
        )
    $\;
    $\mathcal{A}^\text{sym}_\text{IND-CPA}$ wins if $\hat{b} = b$
    \end{algorithm}

    $\mathcal{A}^\text{sym}_\text{IND-CPA}$ perfectly simulates game 2 and wins iff $\mathcal{A}^\text{hy}_\text{IND-CCA}$ wins
    \begin{equation*}
        \alert{
            \epsilon_2 = \epsilon^\text{sym}_\text{IND-CPA}
        }
    \end{equation*}

\end{frame}

\begin{frame}
    \frametitle{Simulate game 2 with OW-CPA adversary}

    \begin{algorithm}[H]
        \SetAlgoLined
        \caption{OW-CPA game against $(E^\text{asym}, D^\text{asym})$}
        $(\text{pk}, \text{sk}) \leftarrow \operatorname{KeyGen}^\text{asym}()$\;
        $\sigma^\ast \leftsample \mathcal{M}^\text{asym}$; 
        $e^\ast \leftsample E^\text{asym}(\text{pk}, \sigma^\ast)$
        \tcp*[h]{truly random coin}\;
        $(m_0, m_1) \leftsample \mathcal{A}^\text{hy}_\text{IND-CCA}(
            \text{pk}, \mathcal{O}^D_1
        )$\;
        $a^\ast \leftsample \mathcal{K}^\text{sym}$;
        $b\leftsample \{0, 1\}$;
        $c^\ast \leftarrow E^\text{sym}_{a^\ast}(m_b)$\;
        $\mathcal{A}^\text{hy}_\text{IND-CCA}(
            \text{pk}, \mathcal{O}^D_1, (e^\ast, c^\ast)
        )$
        \tcp*[h]{discard the output}\;
        
        \alert{Sample a random $\sigma$ from the tape of $H$ or $G$}\;

        $\mathcal{A}^\text{asym}_\text{OW-CPA}$ wins if $\sigma = \sigma^\ast$
    \end{algorithm}

    $\mathcal{A}^\text{asym}_\text{OW-CPA}$ wins if $\mathcal{A}^\text{hy}_\text{IND-CCA}$ queried on $\sigma^\ast$ (aka $\text{QUERY}^\ast$) and the randomly chosen $\sigma$ is the correct one:

    \begin{equation*}
        \alert{
            \epsilon^\text{asym}_\text{OW-CPA} 
            = P[\text{QUERY}^\ast] \cdot \frac{1}{q_H}
        }
    \end{equation*}
\end{frame}

\begin{frame}
    \frametitle{FO 1999, recap}

    \begin{equation*}
        \epsilon^\text{hy}_\text{IND-CCA} 
        \leq q_D2^{-\gamma} 
        + \alert{q_H\epsilon^\text{asym}_\text{OW-CPA} }
        + \epsilon^\text{sym}_\text{IND-CPA} 
    \end{equation*}

    \begin{itemize}
        \item But it's not a KEM?
        \item \alert{Non-tight security}
    \end{itemize}

\end{frame}

\begin{frame}
    \frametitle{Hofheinz, Hovelmanns, Kiltz, 2017}

    \textit{"A modular analysis of the Fujisaki-Okamoto transformation"}

    \begin{itemize}
        \item Tighter security
        \item No need for SKE
        \item IND-CCA KEM
        \item Used by Kyber and McEliece
    \end{itemize}

\end{frame}

\begin{frame}
    \frametitle{Modularity}

    The transformation happens in two steps

    \begin{enumerate}
        \item OW-CPA/IND-CPA PKE to OW-PCVA PKE
        \item OW-PCVA PKE to IND-CCA KEM
    \end{enumerate}

\end{frame}

\begin{frame}
    \frametitle{What is PCVA?}

    In addition to CPA, the adversary can access two more oracles:

    \begin{itemize}
        \item \textbf{Plaintext checking oracle (PCO)} takes a pair of $(m, c)$ and check if they are valid encryption/decryption of each other
        \item \textbf{Ciphertext validation oracle (CVO)} takes a ciphertext $c$ and checks if it is a valid ciphertext
    \end{itemize}

\end{frame}

\begin{frame}
    \frametitle{Vanilla PCO, CVO}

    The vanilla implementations use the secret key to run the decryption routine

    \begin{columns}
        \column{0.5\textwidth}
        \begin{algorithm}[H]
        \SetAlgoLined
        \caption{$\mathcal{O}^\text{CVO}$}
        \KwIn{$\tilde{c}$}
        $\hat{m} \leftarrow D(\text{sk}, c)$\;
        \If{$\hat{m} = \bot$}{
            \Return{$\bot$}\;
        }
        \If{$E(\text{pk}, \hat{m}) \neq \tilde{c}$}{
            \Return{$\bot$}\;
        }
        \Return 1\;
        \end{algorithm}

        \column{0.5\textwidth}
        \begin{algorithm}[H]
        \SetAlgoLined
        \caption{$\mathcal{O}^\text{PCO}$}
        \KwIn{$(\tilde{m}, \tilde{c})$}
        \If{$D(\text{sk}, \tilde{c}) \neq \tilde{m}$}{
            \Return{$\bot$};
        }
        \If{$E(\text{pk}, \tilde{m}) \neq \tilde{c}$}{
            \Return{$\bot$};
        }
        \Return{1}\;
    \end{algorithm}
    \end{columns}
\end{frame}

\begin{frame}
    \frametitle{The OW-PCVA transformation $(E^T, D^T)$}

    Inputs:
    \begin{itemize}
        \item A PKE $(E, D)$
        \item A hash function $G$
    \end{itemize}

    \begin{columns}
        \column{0.5\textwidth}
        \begin{algorithm}[H]
            \SetAlgoLined
            \caption{$E^T$}
            \KwIn{$\text{pk}, m$}
            $r \leftarrow G(m)$\;
            $c \leftarrow E(\text{pk}, m, r)$\;
            \Return{c}\;
        \end{algorithm}

        \column{0.5\textwidth}
        \begin{algorithm}[H]
            \SetAlgoLined
            \caption{$D^T$}
            \KwIn{$\text{sk}, \text{pk}, c$}
            $\hat{m} \leftarrow D(\text{sk}, c)$\;
            $\hat{r} \leftarrow G(\hat{m})$\;
            \If{$E(\text{pk}, \hat{m}, \hat{r}) \neq c$}{
                \Return{$\bot$}\;
            }
            \Return{$\hat{m}$}\;
        \end{algorithm}
    \end{columns}
\end{frame}

\begin{frame}
    \frametitle{OW-PCVA security of $(E^T, D^T)$}

    \begin{alertblock}{Theorem}
        For every OW-PCVA adversary against the T-transformation $(E^T, D^T)$ with advantage $\epsilon^T_\text{OW-PCVA}$ there exists an IND-CPA adversary against the underlying PKE $(E, D)$ with advantage $\epsilon_\text{IND-CPA}$ such that

        \begin{equation*}
            \epsilon^T_\text{OW-PCVA} 
            \leq q_V2^{-\gamma}
            + q_H\delta
            + \frac{1}{\vert\mathcal{M}\vert} 
            + 3\epsilon_\text{IND-CPA}
        \end{equation*}
    \end{alertblock}
\end{frame}

\begin{frame}
    \frametitle{OW-PCVA proof overview}
    Similar strategy to the FO 1999 proof:
    \begin{itemize}
        \item Modify PCO and CVO so that they don't use secret key
        \item Simulate OW-PCVA game using an IND-CPA adversary
    \end{itemize}
\end{frame}

\begin{frame}
    \frametitle{Modified PCO}

    Instead of checking both encryption and decryption, check only encryption

    \begin{columns}
        \column{0.5\textwidth}
        \begin{algorithm}[H]
            \SetAlgoLined
            \caption{$\mathcal{O}^\text{PCO}$}
            \KwIn{$(\tilde{m}, \tilde{c})$}
            \If{$E(\text{pk}, \tilde{m}) \neq \tilde{c}$}{
                \Return{$\bot$}\;
            }
            \If{$D(\text{sk}, \tilde{c}) \neq \tilde{m}$}{
                \Return{$\bot$}\;
            }
            \Return{1}\;
        \end{algorithm}

        \column{0.5\textwidth}
        \begin{algorithm}[H]
            \SetAlgoLined
            \caption{$\mathcal{O}^\text{PCO}_1$}
            \KwIn{$(\tilde{m}, \tilde{c})$}
            \If{$E(\text{pk}, \tilde{m}) \neq \tilde{c}$}{
                \Return{$\bot$}\;
            }
            \Return{1}\;
        \end{algorithm}
    \end{columns}
\end{frame}

\begin{frame}
    \frametitle{Modified CVO}

    Instead of running the decryption routine, check the hash oracle $G$

    \begin{columns}
        \column{0.4\textwidth}
        \begin{algorithm}[H]
            \SetAlgoLined
            \caption{$\mathcal{O}^\text{CVO}$}
            \KwIn{$\tilde{c}$}
            $\hat{m} \leftarrow D(\text{sk}, \tilde{c})$\;
            \If{
                $\hat{m} = \bot$
            }{
                \Return{$\bot$}\;
            }
            \If{
                $E(\text{pk}, \hat{m}) \neq \tilde{c}$
            }{
                \Return{$\bot$}\;
            }
            \Return{$1$}\;
        \end{algorithm}

        \column{0.64\textwidth}
        \begin{algorithm}[H]
            \SetAlgoLined
            \caption{$\mathcal{O}^\text{CVO}_1$}
            \KwIn{$\tilde{c}$}
            \If{
                $\exists (m, r) \in G$ s.t. $E(\text{pk}, m) = \tilde{c}$
            }{
                \Return{$1$}\;
            }
            \Return{$\bot$}
        \end{algorithm}
    \end{columns}
\end{frame}

\begin{frame}
    \frametitle{Game 0: OW-PCVA game}

    \begin{algorithm}[H]
        \SetAlgoLined
        \caption{Game 0}

        $(\text{pk}, \text{sk}) \leftsample \operatorname{KeyGen}()$\;
        $m^\ast \leftsample \mathcal{M}$;
        $c^\ast \leftarrow E^T(\text{pk}, m^\ast)$\;
        $\hat{m} \leftarrow \mathcal{A}^\text{T}_\text{OW-PCVA}(
            \text{pk}, c^\ast, \mathcal{O}^\text{PCO}, \mathcal{O}^\text{CVO}
        )$\;
        $\mathcal{A}^T_\text{OW-PCVA}$ wins if $\hat{m} = m^\ast$
    \end{algorithm}
\end{frame}

\begin{frame}
    \frametitle{Game 1: modify the PCO}

    \begin{algorithm}[H]
        \SetAlgoLined
        \caption{Game 0}

        $(\text{pk}, \text{sk}) \leftsample \operatorname{KeyGen}()$\;
        $m^\ast \leftsample \mathcal{M}$;
        $c^\ast \leftarrow E^T(\text{pk}, m^\ast)$\;
        $\hat{m} \leftarrow \mathcal{A}^\text{T}_\text{OW-PCVA}(
            \text{pk}, c^\ast,
            \alert{\mathcal{O}^\text{PCO}_1},
            \mathcal{O}^\text{CVO}
        )$\;
        $\mathcal{A}^T_\text{OW-PCVA}$ wins if $\hat{m} = m^\ast$
    \end{algorithm}

    \begin{block}{Remark}
        Loss of tightness when decryption error \footnote{
            A PKE is $\delta$-correct if for some fixed keypair and a randomly sampled $m$, $P[D(\text{sk}, E(\text{pk}, m)) \neq m] \leq \delta$
        } happens:

        \begin{equation*}
            \epsilon_0 - \epsilon_1 \leq q_G\delta
        \end{equation*}
    \end{block}
\end{frame}

\begin{frame}
    \frametitle{Game 2: modify the CVO}

    \begin{algorithm}[H]
        \SetAlgoLined
        \caption{Game 0}

        $(\text{pk}, \text{sk}) \leftsample \operatorname{KeyGen}()$\;
        $m^\ast \leftsample \mathcal{M}$;
        $c^\ast \leftarrow E^T(\text{pk}, m^\ast)$\;
        $\hat{m} \leftarrow \mathcal{A}^\text{T}_\text{OW-PCVA}(
            \text{pk}, c^\ast,
            \alert{\mathcal{O}^\text{PCO}_1},
            \alert{\mathcal{O}^\text{CVO}_1}
        )$\;
        $\mathcal{A}^T_\text{OW-PCVA}$ wins if $\hat{m} = m^\ast$
    \end{algorithm}

    \begin{block}{Remark}
        Loss of tightness when $\mathcal{A}$ queried some $\tilde{c}$ without querying $G$

        \begin{equation*}
            \epsilon_1 - \epsilon_2 \leq q_V 2^{-\gamma}
        \end{equation*}
    \end{block}
\end{frame}

\begin{frame}
    \frametitle{Game 3: use a truly random coin}

    \begin{algorithm}[H]
        \SetAlgoLined
        \caption{Game 0}

        $(\text{pk}, \text{sk}) \leftsample \operatorname{KeyGen}()$\;
        $m^\ast \leftsample \mathcal{M}$;
        \alert{
            $r^\ast \leftsample \text{Coin}$;
            $c^\ast \leftarrow E(\text{pk}, m^\ast, r^\ast)$
        }\;
        $\hat{m} \leftarrow \mathcal{A}^\text{T}_\text{OW-PCVA}(
            \text{pk}, c^\ast,
            \alert{\mathcal{O}^\text{PCO}_1},
            \alert{\mathcal{O}^\text{CVO}_1}
        )$\;
        $\mathcal{A}_\text{OW-CPA}$ wins if $\hat{m} = m^\ast$
    \end{algorithm}

    \begin{block}{Remark}
        Loss of tightness when $\mathcal{A}$ queries $G$ on $m^\ast$

        \begin{equation*}
            \epsilon_2 - \epsilon_3 \leq P[\text{QUERY}^\ast]
        \end{equation*}
    \end{block}
\end{frame}

\begin{frame}
    \frametitle{Simulate game 3 with OW-CPA adversary}

    Game 3 can be perfectly simulated by an OW-CPA adversary against the underlying PKE $(E, D)$:

    \begin{equation*}
        \alert{\epsilon_3 = \epsilon_\text{OW-CPA}}
    \end{equation*}

    The OW-CPA advantage can be directly translated into IND-CPA advantage with the following "well-known results":

    \begin{block}{Theorem}
        For every IND-CPA adversary with advantage $\epsilon_\text{IND-CPA}$ there exists an OW-CPA adversary with advantage $\epsilon_\text{OW-CPA}$ such that

        \begin{equation*}
            \epsilon_\text{OW-CPA} 
            = \frac{1}{\vert\mathcal{M}\vert} + \epsilon_\text{IND-CPA}
        \end{equation*}
    \end{block}
\end{frame}

\begin{frame}
    \frametitle{Simulate game 3 with IND-CPA adversary}

    We can build $\mathcal{A}_\text{IND-CPA}$ that:
    \begin{itemize}
        \item Sample random $m_0, m_1$
        \item Check the hash function tape for matching $m_b$
    \end{itemize}

    \begin{algorithm}[H]
        \SetAlgoLined
        \caption{IND-CPA game against $(E, D)$}

        $(\text{pk}, \text{sk}) \leftsample \operatorname{KeyGen}()$\;
        $(m_0, m_1) \leftsample \mathcal{M}$\footnote{
            We omit nuance about sampling $m_0, m_1$ randomly while making sure that they are distinct
        }\;
        $b \leftsample \{0,1\}$; $c^\ast = E(\text{pk}, m_b)$\;
        $\hat{m} \leftarrow \mathcal{A}^T_\text{OW-PCVA}(
            \text{pk}, c^\ast, \mathcal{O}^\text{PCO}_1, \mathcal{O}^\text{CVO}_1
        )$\;
        $\hat{b} \leftarrow \alert{\operatorname{CheckTape()}}$\;
        $\mathcal{A}_\text{IND-CPA}$ wins if $\hat{b} = b$\;
    \end{algorithm}
\end{frame}

\begin{frame}
    \frametitle{CheckTape()}

    If $\exists(m,r) \in G$ such that $m = m_0$ or $m = m_1$, then set $\hat{b} = 0$ or $\hat{b} = 1$ accordingly.

    If no such record exists, return a blind guess $\hat{b} \leftsample \{0,1\}$

    \begin{equation*}
        P[\hat{b} = b] = P[\text{QUERY}^\ast]
            + (1 - P[\text{QUERY}^\ast])\frac{1}{2}
    \end{equation*}

    Which implies

    \begin{equation*}
        \alert{\epsilon_\text{IND-CPA} = \frac{1}{2}P[\text{QUERY}^\ast]}
    \end{equation*}
\end{frame}

\begin{frame}
    \frametitle{IND-CCA KEM}

    \begin{table}
        \centering
        \begin{tabular}{|c|c|c|}
        \hline
        & explicit rejection & implicit rejection \\
        \hline
        PKE is IND-CPA & $U^\bot$ & $U^{\not\bot}$ \\
        PKE is OW-CPA & $U^\bot_m$ & $U^{\not\bot}_m$ \\
        \hline
        \end{tabular}
        \caption{KEM transformations}
    \end{table}
\end{frame}

\begin{frame}
    \frametitle{$U^\bot$ implementation}

    $H$ is another hash function

    \begin{columns}
        \column{0.5\textwidth}
        \begin{algorithm}[H]
            \SetAlgoLined
            \caption{$U^\bot$ Encap}
            \KwIn{$\text{pk}$}
            $m \leftsample \mathcal{M}$\;
            $c \leftarrow E^T(\text{pk}, m)$\;
            $K \leftarrow H(m, c)$\;
            \Return{$(c, K)$}\;
        \end{algorithm}

        \column{0.5\textwidth}
        \begin{algorithm}[H]
            \SetAlgoLined
            \caption{$U^\bot$ Decap}
            \KwIn{$\text{sk}, c$}
            \KwOut{Shared secret}
            $m \leftarrow D^T(\text{sk}, c)$\;
            \If{
                $m = \bot$
            }{
                \Return{$\bot$};
            }
            \Return{$H(m, c)$}\;
        \end{algorithm}
    \end{columns}
\end{frame}

\begin{frame}
    \frametitle{$U^\bot$ security}

    For every IND-CCA adversary against $U^\bot$ with advantage $\epsilon^{U^\bot}_\text{IND-CCA}$, there exists an OW-PCVA adversary against $(E^T, D^T)$ with advantage $\epsilon^T_\text{OW-PCVA}$ such that

    $$
    \epsilon^{U^\bot}_\text{IND-CCA} \leq \epsilon^T_\text{OW-PCVA}
    $$

\end{frame}

\begin{frame}
    \frametitle{Simulate decapsulation oracle}

    \begin{block}{Goal}
        If the query $\tilde{c}$ is a valid ciphertext that decrypts to $\tilde{m}$, $\mathcal{O}^D$ should return $H(\tilde{m}, \tilde{c})$
    \end{block}

    \begin{block}{Strategy}
        \begin{itemize}
            \item Make both $H$ and $\mathcal{O}^D$ stateful
            \item Use PCO and CVO to "decrypt" and check integrity
        \end{itemize}    
    \end{block}
\end{frame}

\begin{frame}
    \frametitle{\textit{Patched} hash and decap oracle}

    $\mathcal{O}^D_1$ keeps track of past queries $(\tilde{c}, \tilde{K})$

    \begin{columns}
        \column{0.5\textwidth}
        \begin{algorithm}[H]
            \SetAlgoLined
            \caption{$H_1$}
            \KwIn{$(\tilde{m}, \tilde{c})$}
            \If{$\exists (\tilde{m}, \tilde{c}, \tilde{K}) \in H_1$}{
                \Return{$\tilde{K}$}\;
            }
            $\tilde{K} \leftsample \{0, 1\}^\ast$\;

            \If{
                $\mathcal{O}^\text{PCO}(\tilde{m}, \tilde{c}) \neq \bot$
            }{
                Append $(\tilde{c}, \tilde{K})$ to $\mathcal{O}^D$
            }

            \Return{$\tilde{K}$}\;
        \end{algorithm}

        \column{0.5\textwidth}
        \begin{algorithm}[H]
            \SetAlgoLined
            \caption{$\mathcal{O}^D_1$}
            \KwIn{$\tilde{c}$}
            \If{
                $(\tilde{c}, \tilde{K}) \in \mathcal{O}^D_1$
            }{
                \Return{$\tilde{K}$}\;
            }
            \If{
                $\mathcal{O}^\text{CVO}(\tilde{c}) = \bot$
            }{
                \Return{$\bot$}\;
            }
            $\tilde{K} \leftsample \{0, 1\}^\ast$\;
            Append $(\tilde{c}, \tilde{K})$ to $\mathcal{O}^D$\;
            \Return{$\tilde{K}$}\;
        \end{algorithm}
    \end{columns}

    Patched oracles behave exactly like their vanilla counterparts
\end{frame}

\begin{frame}
    \frametitle{Game 0: KEM IND-CCA}

    \begin{algorithm}[H]
        \SetAlgoLined
        \caption{Game 0: KEM IND-CCA}
        $(\text{pk}, \text{sk}) \leftsample \operatorname{KeyGen}()$\;
        $(c^\ast, K_0) \leftsample E^{U^\bot}(\text{pk})$;
        $K_1 \leftsample \{0, 1\}^\ast$\;
        $b \leftsample\{0,1\}$;
        $K^\ast \leftarrow K_b$\;
        $\hat{b} \leftsample \mathcal{A}^{U^\bot}_\text{IND-CCA}(
            \text{pk}, c^\ast, K^\ast, \mathcal{O}^D, H
        )$\;
        $\mathcal{A}^{U^\bot}_\text{IND-CCA}$ wins if $\hat{b} = b$
    \end{algorithm}
\end{frame}

\begin{frame}
    \frametitle{Game 1: Use patched oracles}

    \begin{algorithm}[H]
        \SetAlgoLined
        \caption{Game 1: with patched oracles}
        $(\text{pk}, \text{sk}) \leftsample \operatorname{KeyGen}()$\;
        $(c^\ast, K_0) \leftsample E^{U^\bot}(\text{pk})$;
        $K_1 \leftsample \{0, 1\}^\ast$\;
        $b \leftsample\{0,1\}$;
        $K^\ast \leftarrow K_b$\;
        $\hat{b} \leftsample \mathcal{A}^{U^\bot}_\text{IND-CCA}(
            \text{pk}, c^\ast, K^\ast, 
            \alert{\mathcal{O}^D_1}, 
            \alert{H_1}
        )$\;
        $\mathcal{A}^{U^\bot}_\text{IND-CCA}$ wins if $\hat{b} = b$
    \end{algorithm}

    \begin{block}{Remark}
        There is no difference between game 0 and game 1

        \begin{equation*}
            \epsilon_0 = \epsilon_1
        \end{equation*}
    \end{block}
\end{frame}

\begin{frame}
    \frametitle{Game 2: Use truly random $K^\ast$}

    \begin{algorithm}[H]
        \SetAlgoLined
        \caption{Game 2: unwinnable game}
        $(\text{pk}, \text{sk}) \leftsample \operatorname{KeyGen}()$\;
        $(c^\ast, K_0) \leftsample E^{U^\bot}(\text{pk})$\;
        \alert{$K^\ast \leftsample \{0, 1\}^\ast$}\;
        $b \leftsample\{0,1\}$\;
        $\hat{b} \leftsample \mathcal{A}^{U^\bot}_\text{IND-CCA}(
            \text{pk}, c^\ast, K^\ast, 
            \alert{\mathcal{O}^D_1}, 
            \alert{H_1}
        )$\;
        $\mathcal{A}^{U^\bot}_\text{IND-CCA}$ wins if $\hat{b} = b$
    \end{algorithm}

    \begin{block}{Remark}
        Game 2 and game 1 diverge when $\mathcal{A}^{}_\text{IND-CCA}$ queries $H$ on $(m^\ast, c^\ast)$

        \begin{equation*}
            \alert{\epsilon_1 - \epsilon_2 \leq P[\text{QUERY}^\ast]}
        \end{equation*}

        Also, game 2 is unwinnable: $\epsilon_2 = 0$
    \end{block}
\end{frame}

\begin{frame}
    \frametitle{Simulate game 2 with OW-PCVA adversary}

    \begin{algorithm}[H]
        \SetAlgoLined
        \caption{OW-PCVA game}
        $(\text{pk}, \text{sk}) \leftsample \operatorname{KeyGen}()$\;
        $m^\ast \leftsample \mathcal{M}$;
        $c^\ast \leftarrow E^T(\text{pk}, m^\ast)$\;
        $K^\ast \leftsample \{0,1\}^\ast$\;
        $\hat{b} \leftarrow \mathcal{A}^{U^\bot}_\text{IND-CCA}(
            \text{pk}, c^\ast, K^\ast, \mathcal{O}^D_1, H_1
        )$\;
        $\hat{m} \leftarrow \operatorname{CheckTape()}$\;
        $\mathcal{A}^T_\text{OW-PCVA}$ wins if $\hat{m} = m^\ast$
    \end{algorithm}

    \begin{block}{Remark}
        $\mathcal{A}^T_\text{OW-PCVA}$ wins if $\mathcal{A}^{U^\bot}_\text{IND-CCA}$ queries on $m^\ast$

        \begin{equation*}
            \alert{P[\text{QUERY}^\ast] = \epsilon^T_\text{OW-PCVA}}
        \end{equation*}
    \end{block}

\end{frame}

\end{document}