\documentclass{article}
\usepackage[margin=1in,letterpaper]{geometry}
\usepackage{amsmath,amsfonts,amssymb,amsthm}

% For source code
\usepackage{listings}

% Algorithms and pseudocode
\usepackage[ruled,vlined]{algorithm2e}
\usepackage{algpseudocode}

% Common shortcuts
\newcommand{\round}[1]{\lfloor {#1} \rceil}
\newcommand{\norm}[1]{\Vert {#1} \Vert}
\newcommand{\var}[1]{\operatorname{Var}[{#1}]}

% Environments: definitions, theorems, propositions, corollaries, lemmas
%    Theorems, propositions, and definitions are numbered within the section
%    Corollaries are numbered within the theorem, though they are rarely used
\newtheorem{definition}{Definition}[section]
\newtheorem{theorem}{Theorem}[section]
\newtheorem*{remark}{Remark}
\newtheorem{corollary}{Corollary}[theorem]
\newtheorem{proposition}{Proposition}[section]
\newtheorem{lemma}{Lemma}[theorem]


\title{CO 789, Homework 1}
\author{Ganyu (Bruce) Xu (g66xu)}
\date{Fall 2023}

\begin{document}
%%%% TITLE %%%%%
% \maketitle

\section*{Question 4}

\subsection*{4.1}
We show equivalence by showing that having an oracle solver for one problem allows us to solve the other problem.

In the forward direction, let $G \in \mathbb{F}^{n \times k}_2$ but a generator matrix, $c \in \mathbb{F}^n_2$ be some partially corrupted codeword, and $\mathcal{O}^S$ be the syndrome decoding oracle. If we can find a corresponding parity-check matrix $H$, then $y \leftarrow Hc \in \mathbb{F}_2^{n-k}$ is a syndrome. We can then feed $H, y$ to $\mathcal{O}^S$ and obtain the error term $e$. Solving the linear system $Gm = c - e$ allows us to recover the message $m$.

To compute the parity check matrix, we assume that the generator matrix $G$ has the following form:

$$
G = \begin{bmatrix}
    I_k \\
    G^\prime
\end{bmatrix}
$$

where $G^\prime \in \mathbb{F}_2^{(n-k) \times k}$. This is a reasonable assumption because if $G$ does not have this form, we can column-reduce $G$ to have this form. Column-reducing $G$ is guaranteed to produce $I_k$ on the top $G$ has column rank $k$ (otherwise the code will map distinct messages onto the same codeword, which cannot happen). Column-reducing $G$ is equivalent to right-multiplication by an invertible matrix, which does not affect the set of codewords in the code $\mathcal{C}$.

Let $H = [-G^\prime \mid I_{n-k}]$, then $H$ is a rank-$(n-k)$ matrix such that $HG = 0$, meaning that $H$ is a parity check matrix that corresponds with $G$.

$$ $$

In the backward direction, let $H \in \mathbb{F}_2^{(n-k) \times n}$ be a parity check matrix, $y \in \mathbb{F}_2^{n-k}$ be a syndrome, and $\mathcal{O}^C$ be a codeword decoding oracle. We can row reduce $H$ to have the form $H = [I_{n-k} \mid H_0]$ where $H_0 \in \mathbb{F}_2^{(n-k) \times k}$. Since row reducing $H$ is equivalent to left multiplication by an invertible matrix, the row-reduced $[I_{n-k} \mid H_0]$ is also a parity check matrix for the same code.

Let $G$ be as follows:

\begin{equation*}
    G = \begin{bmatrix}
        -H_0 \\ I_k
    \end{bmatrix}
\end{equation*}

Then $G \in \mathbb{F}_2^{n \times k}$ is a rank-$k$ matrix such that $HG = 0$. In other words, $H$ is the parity check matrix of a linear code generated by $G$. If we solve the linear system $Hz = y$ for $z$, then $z$ is some partially corrupted codeword in the linear code. Let $m \leftarrow \mathcal{O}^C(G, z)$. By the definition of the codeword decoding problem $e = z - Gm$ is the error term.

\subsection*{4.2}
Again, we will show equivalence by showing that a solver for one problem allows us to solve the other problem.

In the forward direction, let $A \in \mathbb{F}_2^{n \times k} = PGS$ and $c \in \mathbb{F}_2^n = Am + e$ be what they are in the McEliece codeword decoding problem. Let $\mathcal{O}^S$ be a McEliece syndrome decoding oracle. If we can find the corresponding $H_0$ such that $[I_{n-k} \mid H_0] = SHP$ for some parity check matrix $H$, then we can compute $y \leftarrow [I_{n-k} \mid H_0]c$, feed $H_0, y$ to $\mathcal{O}^S$ to obtain the error term $e$, then solve the linear system $Am = c - e$ to recover the message $m$, which is the solution to the McEliece codeword decoding problem.

To compute $H_0$, we first column-reduce $A$ such that

$$
A = \begin{bmatrix}
    I_k \\ A^\prime
\end{bmatrix}
$$

where $A^\prime \in \mathbb{F}_2^{(n-k) \times k}$. Column-reducing $A$ is equivalent to right multiplication by an invertible matrix, so the column-reduced $A$ still corresponds to the same generator matrix $G$. In addition, both $P, S$ are invertible, so $A$ is guaranteed to have rank $k$.

Let $\tilde{H} = [-A^\prime \mid I_{n-k}]$, then $\tilde{H}A = \tilde{H}PGS = 0$, which means that $H = \tilde{H}P$ is a parity check matrix for $G$. If we then row reduce $\tilde{H} = HP^{-1}$:

\begin{equation*}
    \tilde{S}\tilde{H} = \tilde{S}HP^{-1} = [I_{n-k} \mid H_0]
\end{equation*}

Since $\tilde{S}$ is invertible and $P^{-1}$ is also a permutation, $H_0$ is indeed the matrix used in the McEliece syndrome decoding.

$$ $$

In the other direction, let $H_0 \in \mathbb{F}_2^{(n-k) \times k}$ and $y \in \mathbb{F}_2^{n-k}$ be what they are in the McEliece syndrome decoding problem. Let $\mathcal{O}^C$ be a McEliece codeword decoding oracle. Let $\tilde{H} = [I_{n-k} \mid H_0]$, then solve the linear system $\tilde{H}z = y$ for $z$.

Let $\tilde{A}$ be as follows:

\begin{equation*}
    \tilde{A} = \begin{bmatrix}
        -H_0 \\ I_k
    \end{bmatrix}
\end{equation*}

Then $\tilde{H}\tilde{A} = 0$. By the definition of the McEliece syndrome decoding problem, we know that $\tilde{H} = SHP$ for some permutation $P$ and some invertible matrix $S$, which means that $SHP\tilde{A} = 0$. Since $\tilde{A}$ has k-rank and $P$ is invertible, $P\tilde{A}$ has rank-k. In other words, $P\tilde{A} = G$ for some generator matrix $G$ that corresponds with $H$. Re-arranging this equation: $\tilde{A} = P^{-1}GI_k$ is a valid McEliece codeword decoding public key.

Let $m \leftarrow O^C(\tilde{A}, z)$, then $e = z - Gm$ is the error term and the solution to the McEliece syndrome decoding problem.

\end{document}