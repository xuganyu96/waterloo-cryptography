\documentclass{article}
\usepackage[margin=1in,letterpaper]{geometry}
\usepackage{amsmath,amsfonts,amssymb,amsthm}

% For source code
\usepackage{listings}

% Algorithms and pseudocode
\usepackage[linesnumbered, ruled, vlined]{algorithm2e}
\usepackage{algpseudocode}

% Common shortcuts
\newcommand{\round}[1]{\lfloor {#1} \rceil}
\newcommand{\Norm}[1]{\Vert {#1} \Vert}
\newcommand{\norm}[1]{\vert {#1} \vert}
\newcommand{\var}[1]{\operatorname{Var}[{#1}]}

% Environments: definitions, theorems, propositions, corollaries, lemmas
%    Theorems, propositions, and definitions are numbered within the section
%    Corollaries are numbered within the theorem, though they are rarely used
\newtheorem{definition}{Definition}
\newtheorem{theorem}{Theorem}
\newtheorem*{remark}{Remark}
\newtheorem{corollary}{Corollary}[theorem]
\newtheorem{proposition}{Proposition}
\newtheorem{lemma}{Lemma}[theorem]


\title{CO 789, Homework 5}
\author{Ganyu (Bruce) Xu (g66xu)}
\date{Winter 2024}

\begin{document}
%%%% TITLE %%%%%
% \maketitle

\section*{Question 2}

\subsection*{(1)}
Here is the decryption algorithm

\begin{algorithm}
    \caption{Compressed McEliece decryption}
    \KwIn{$c \in \mathbb{F}_2^{n - l}$, $\text{sk} = (P, \mathcal{C}, S)$}

    Pad the low-order bits of $c$ with 0's such that $
        c^\prime = [c, 0, 0, \ldots, 0] \in \mathbb{F}^n
    $\;
    $c^{\prime\prime} \leftarrow P^{-1}c^\prime$\;
    $\hat{m} \leftarrow \mathcal{C}.\text{decode}(c^{\prime\prime})$\;
    $\hat{m} \leftarrow S^{-1}\hat{m}$\;
    \Return{$\hat{m}$}\;
\end{algorithm}

Among the high-order bits of $c^{\prime\prime}$, there are exactly $t - l$ bits of error, introduced by the sampled error term $e$. Among the low-order $l$ bits of $c^{\prime\prime}$, there are up to $l$ bits of errors since the padded 0's are blind guesses. Therefore, there are up to $t$ bits of errors in $c^{\prime\prime}$, and since $t \leq \frac{d-1}{2}$, the $(n, k, d)$-code is guaranteed to correct the error and recover the true $m$.

\subsection*{(2)}
The vanilla McEliece encryption scheme outputs the entire (partially corrupted codeword) $c$, which takes $n$ bits. This compressed scheme discarded the low-order $l$ bits, so the ciphertext takes $n - l$ bits, which saves space by a factor of $\frac{l}{n}$

\subsection*{(3)}
We already know that the McEliece encryption scheme is not IND-CPA. Instead, we will estimate the difficulty of breaking the encryption scheme by computing the probability of correctly decrypting some ciphertext without using the secret key. Specifically, since $c = Am + e$ where $A \in \mathbb{F}^{n \times k}$ is an overdetermined linear system, if the adversary can recover $e$, then it can recover $m$.

In the un-compressed McEliece scheme, the encryption routine corrupts exactly $t$ out of $n$ bits. There are a total of $\binom{n}{t}$ possible error terms to choose from. Without knowing additional information, the adversary can do no better than a blind guess:

\begin{equation*}
    \epsilon_0 = \frac{1}{\binom{n}{t}}
\end{equation*}

In the compressed scheme, the encryption routine corrupts $t-l$ out of the high-order $n-l$ bits. The low-order $l$ bits are blind guesses. There are a total of $\binom{n-l}{t-l} \cdot 2^l$ possible error values to choose from:

\begin{equation*}
    \epsilon_1 = \frac{1}{\binom{n-l}{t-l} \cdot 2^l}
\end{equation*}

Observe that

$$
\begin{aligned}
    \frac{\epsilon_0}{\epsilon_1} 
    &= \frac{\binom{n-l}{t-l} \cdot 2^l}{\binom{n}{t}} \\
    &= \frac{
        t \cdot (t-1) \cdot \ldots \cdot (t - l + 1)
    }{
        n \cdot (n-1) \cdot \ldots \cdot (n - l + 1)
    } \cdot 2^l
\end{aligned}
$$

Since $t = \frac{d-1}{2} \leq \frac{d}{2} < \frac{n}{2}$, we know $\frac{t}{n} \leq \frac{1}{2}$. Furthermore, we claim without proof that if $0 < a < b$ then $\frac{a}{b} \geq \frac{a-1}{b-1}$, which means

$$
\frac{1}{2} > \frac{t}{n} > \frac{t-1}{n-1} > \ldots > \frac{t-l+1}{n-l+1}
$$

Therefore:

$$
\frac{\epsilon_0}{\epsilon_1} = \frac{
    t \cdot (t-1) \cdot \ldots \cdot (t - l + 1)
}{
    n \cdot (n-1) \cdot \ldots \cdot (n - l + 1)
} \cdot 2^l \leq 1
$$

This means that a blindly-guessing adversary against the compressed scheme has higher advantage than a blindly-guessing adversary against the vanilla scheme. In other words, the compressed scheme is less secure.

\subsection*{(4)}
For a linear amount of reduction in ciphertext size, we lose an exponential amount of security. This tradeoff is not worth it.
\end{document}