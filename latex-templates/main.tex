\documentclass{article}
\usepackage[margin=1in,letterpaper]{geometry}
\usepackage{amsmath,amsfonts,amssymb,amsthm}

% For source code
\usepackage{listings}

% Algorithms and pseudocode
% \usepackage[linesnumbered,ruled,vlined]{algorithm2e}
\usepackage{algorithm}
\usepackage{algpseudocode}
\usepackage{hyperref}

% Custom commands
\usepackage{crypto-primitives}
\usepackage{kyber-algos}

% Environments: definitions, theorems, propositions, corollaries, lemmas
%    Theorems, propositions, and definitions are numbered within the section
%    Corollaries are numbered within the theorem, though they are rarely used
\newtheorem{definition}{Definition}[section]
\newtheorem{theorem}{Theorem}[section]
\newtheorem*{remark}{Remark}
\newtheorem{corollary}{Corollary}[theorem]
\newtheorem{proposition}{Proposition}[section]
\newtheorem{lemma}{Lemma}[theorem]


\title{
    Document title
}
\author{
    Author 1
}
% Leave the date field empty to display the date of compilation
% \date{}

\begin{document}
%%%% TITLE %%%%%
\maketitle

\section{Kyber and ML-KEM }
CRYSTALS-Kyber \cite{bos2018crystals} is an \texttt{IND-CCA} secure key encapsulation mechanism whose security is based on the hardness of the Module Learning with Error (MLWE) problem. It is submitted to NIST's ``Pots-Quantum Cryptography'' contest, where it advanced to the third round \cite{avanzi2019crystals}.

A modified version was standardized by NIST \cite{key2023mechanism} and renamed to ``ML-KEM''.

\kybercpapkekeygen
\kybercpapkeenc
\kybercpapkedec
\kyberccakemkeygen
\kyberccakemencap
\kyberccakemdecap



\bibliographystyle{alpha}
\bibliography{references}

\end{document}