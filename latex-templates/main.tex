\documentclass{article}
\usepackage[margin=1in,letterpaper]{geometry}
\usepackage{amsmath,amsfonts,amssymb,amsthm}

% For source code
\usepackage{listings}

% Algorithms and pseudocode
% \usepackage[linesnumbered,ruled,vlined]{algorithm2e}
\usepackage{algorithm}
\usepackage{algpseudocode}

\usepackage{multicol}

% Common shortcuts
\newcommand{\round}[1]{\left\lfloor {#1} \right\rceil}
\newcommand{\abs}[1]{\left\lvert {#1} \right\rvert}
\newcommand{\norm}[1]{\left\lvert {#1} \right\rvert}
\newcommand{\Norm}[1]{\left\lVert {#1} \right\rVert}
\newcommand{\monospace}{\texttt}
\newcommand{\var}[1]{\operatorname{Var}[{#1}]}
% \newcommand{\leftsample}{\overset{{\scriptscriptstyle\$}}{\leftarrow}}
\newcommand{\leftsample}{\stackrel{\$}{\leftarrow}}
\newcommand{\pke}{\monospace{PKE}}
\newcommand{\keygen}{\monospace{KeyGen}}
\newcommand{\encrypt}{\monospace{E}}
\newcommand{\decrypt}{\monospace{D}}
\newcommand{\kem}{\monospace{KEM}}
\newcommand{\encap}{\monospace{Encap}}
\newcommand{\decap}{\monospace{Decap}}
\newcommand{\mac}{\monospace{MAC}}
\newcommand{\sign}{\monospace{S}}
\newcommand{\verify}{\monospace{V}}
\newcommand{\pk}{\monospace{pk}}
\newcommand{\sk}{\monospace{sk}}
\newcommand{\llbrack}{[\![}
\newcommand{\rrbrack}{]\!]}
\newlength{\wdth}
\newcommand{\strike}[1]{\settowidth{\wdth}{#1}\rlap{\rule[.5ex]{\wdth}{.4pt}}#1}

% Environments: definitions, theorems, propositions, corollaries, lemmas
%    Theorems, propositions, and definitions are numbered within the section
%    Corollaries are numbered within the theorem, though they are rarely used
\newtheorem{definition}{Definition}[section]
\newtheorem{theorem}{Theorem}[section]
\newtheorem*{remark}{Remark}
\newtheorem{corollary}{Corollary}[theorem]
\newtheorem{proposition}{Proposition}[section]
\newtheorem{lemma}{Lemma}[theorem]


\title{CO 789, Homework 1}
\author{Ganyu (Bruce) Xu (g66xu)}
\date{Fall 2023}

\begin{document}
%%%% TITLE %%%%%
\maketitle

\section{First section}
\begin{definition}[Lattice]\label{lattice-def}
A lattice is a discrete subgroup of $\mathbb{R}^n$
\end{definition}

\begin{theorem}[Minkowski's bound]
    let $\mathcal{L}(B)$ be a full-rank lattice with basis $B \in \mathbb{R}^{n \times n}$, and $B^\ast = [\mathbf{b}_1^\ast, \mathbf{b}_2^\ast, \ldots, \mathbf{b}_n^\ast]$ be the Gram-Schmidt orthogonalization of $B$, then
    \begin{equation}
        \lambda_1(\mathcal{L}(B)) \geq \min_{1 \leq i \leq n}\norm{\mathbf{b}_i^\ast}
    \end{equation}
\end{theorem}

\begin{algorithm}
    \caption{An algorithm with caption}\label{alg:cap}
    \begin{algorithmic}
        \Require $n \geq 0$
        \Ensure $y = x^n$
        \State $y \gets 1$
        \State $X \gets x$
        \State $N \gets n$
        \While{$N \neq 0$}
            \If{$N$ is even}
                \State $X \gets X \times X$
                \State $N \gets \frac{N}{2}$  \Comment{This is a comment}
            \ElsIf{$N$ is odd}
                \State $y \gets y \times X$
                \State $N \gets N - 1$
            \EndIf
        \EndWhile
    \end{algorithmic}
\end{algorithm}

Here is some citation\cite{fujisaki1999secure}

\bibliographystyle{plain}
\bibliography{./references.bib}

\end{document}