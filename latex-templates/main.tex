\documentclass[runningheads]{llncs}

\usepackage[T1]{fontenc}
\usepackage{graphicx}
\usepackage{hyperref}
\usepackage{color}
\renewcommand\UrlFont{\color{blue}\rmfamily}
\urlstyle{rm}
\usepackage{amsmath,amsfonts,amssymb}
\usepackage{listings}
% \usepackage[linesnumbered,ruled,vlined]{algorithm2e}
\usepackage{algorithm}
\usepackage{algpseudocode}
\begin{document}
\title{Document title}

% If the paper title is too long for the running head, you can set
% an abbreviated paper title here
%\titlerunning{Abbreviated paper title}

\author{
    Ganyu Xu\inst{1}
    % First Author\inst{1}\orcidID{0000-1111-2222-3333} \and
    % Second Author\inst{2,3}\orcidID{1111-2222-3333-4444} \and
    % Third Author\inst{3}\orcidID{2222--3333-4444-5555}
}
% First names are abbreviated in the running head.
% If there are more than two authors, 'et al.' is used.
%
\authorrunning{G. Xu et al.}

\institute{
    University of Waterloo, Waterloo, Ontario, Canada
    \email{g66xu@uwaterloo.ca}
}

\maketitle 

\begin{abstract}
    The abstract should briefly summarize the contents of the paper in
    150--250 words.

    \keywords{
        First keyword  
        \and Second keyword
        \and Another keyword.
    }
\end{abstract}

\section{First Section}
\subsection{A Subsection Sample}
Please note that the first paragraph of a section or subsection is
not indented. The first paragraph that follows a table, figure,
equation etc. does not need an indent, either.

Subsequent paragraphs, however, are indented.

\subsubsection{Sample Heading (Third Level)} Only two levels of
headings should be numbered. Lower level headings remain unnumbered;
they are formatted as run-in headings.

Some reference to algorithm in Figure \ref{fig:some-algorithm}.

\begin{figure}[H]
    \centering
    \begin{minipage}[t]{0.5\textwidth}
    \begin{algorithm}[H]
        \caption{Some algorithm}
        \begin{algorithmic}[1]
            \State $x \leftarrow 1$
        \end{algorithmic}
    \end{algorithm}
    \end{minipage}
    \caption{Figure containing algorithm}\label{fig:some-algorithm}
\end{figure}

\paragraph{Sample Heading (Fourth Level)}
The contribution should contain no more than four levels of
headings. Table~\ref{tab1} gives a summary of all heading levels.

\begin{table}
\caption{Table captions should be placed above the tables.}\label{tab1}
    \begin{tabular}{|l|l|l|}
    \hline
    Heading level &  Example & Font size and style\\
    \hline
    Title (centered) &  {\Large\bfseries Lecture Notes} & 14 point, bold\\
    1st-level heading &  {\large\bfseries 1 Introduction} & 12 point, bold\\
    2nd-level heading & {\bfseries 2.1 Printing Area} & 10 point, bold\\
    3rd-level heading & {\bfseries Run-in Heading in Bold.} Text follows & 10 point, bold\\
    4th-level heading & {\itshape Lowest Level Heading.} Text follows & 10 point, italic\\
    \hline
    \end{tabular}
\end{table}


\noindent Displayed equations are centered and set on a separate
line.
\begin{equation}
x + y = z
\end{equation}

\begin{theorem}
This is a sample theorem. The run-in heading is set in bold, while
the following text appears in italics. Definitions, lemmas,
propositions, and corollaries are styled the same way.
\end{theorem}
%
% the environments 'definition', 'lemma', 'proposition', 'corollary',
% 'remark', and 'example' are defined in the LLNCS documentclass as well.
%
\begin{proof}
Proofs, examples, and remarks have the initial word in italics,
while the following text appears in normal font.
\end{proof}

For citations of references, we prefer the use of square brackets and consecutive numbers. Citations using labels or the author/year
convention are also acceptable. The following bibliography provides
a sample reference list with entries for journal
articles~\cite{wegman1981new}. Multiple citations are grouped
\cite{wegman1981new,bernstein2018towards,bellare1998relations}.

\begin{credits}
\subsubsection{\ackname} A bold run-in heading in small font size at the end of the paper is
used for general acknowledgments, for example: This study was funded
by X (grant number Y).

\subsubsection{\discintname}
It is now necessary to declare any competing interests or to specifically
state that the authors have no competing interests. Please place the
statement with a bold run-in heading in small font size beneath the
(optional) acknowledgments\footnote{If EquinOCS, our proceedings submission
system, is used, then the disclaimer can be provided directly in the system.},
for example: The authors have no competing interests to declare that are
relevant to the content of this article. Or: Author A has received research
grants from Company W. Author B has received a speaker honorarium from
Company X and owns stock in Company Y. Author C is a member of committee Z.
\end{credits}

% ---- Bibliography ----
%
% BibTeX users should specify bibliography style 'splncs04'.
% References will then be sorted and formatted in the correct style.
%
\bibliographystyle{splncs04}
\bibliography{references.bib}
\end{document}
