\documentclass{article}
\usepackage[margin=1in,letterpaper]{geometry}
\usepackage{amsmath,amsfonts,amssymb,amsthm}

% For source code
\usepackage{listings}

% Algorithms and pseudocode
% \usepackage[linesnumbered,ruled,vlined]{algorithm2e}
\usepackage{algorithm}
\usepackage{algpseudocode}
\usepackage{hyperref}

% Custom commands
\usepackage{crypto-primitives}
% Quantum notations
\newcommand{\qbra}[1]{\left\langle {#1} \right\rvert}
\newcommand{\qket}[1]{\left\lvert {#1} \right\rangle}

% Environments: definitions, theorems, propositions, corollaries, lemmas
%    Theorems, propositions, and definitions are numbered within the section
%    Corollaries are numbered within the theorem, though they are rarely used
\newtheorem{definition}{Definition}[section]
\newtheorem{theorem}{Theorem}[section]
\newtheorem*{remark}{Remark}
\newtheorem{corollary}{Corollary}[theorem]
\newtheorem{proposition}{Proposition}[section]
\newtheorem{lemma}{Lemma}[theorem]


\title{
    Homework 1
}
\author{
    Ganyu Xu
}
% Leave the date field empty to display the date of compilation
% \date{}

\begin{document}
%%%% TITLE %%%%%
\maketitle

\section{Distinguish quantum states}
\subsection*{(a)}
Let $\psi_0 = \qket{1}$ and $\psi_1 = \frac{1}{\sqrt{2}}\qket{0} - \frac{1}{\sqrt{2}}\qket{1}$. I claim without proof that \textbf{an optimal distinguisher can be achieved using $R(\theta)$} for some rotation $\theta$. Without loss of generality, we define the distinguisher to be $\hat{b} \leftarrow D(\phi_b)$ where $b \leftsample \{0,1\}$ and $\hat{b}$ is $D$'s guess at the choice of quantum state.

After the rotation, the quantum states are as follows:

\begin{equation*}
    \begin{aligned}
        R \cdot \psi_0 &= \begin{bmatrix}
            \cos(\theta) & -\sin(\theta) \\
            \sin(\theta) & \cos(\theta)
        \end{bmatrix} \cdot \begin{bmatrix}
            0 \\ 1
        \end{bmatrix} \\
        &= \begin{bmatrix}
            -\sin(\theta) \\ \cos(\theta)
        \end{bmatrix}
    \end{aligned}
\end{equation*}

\begin{equation*}
    \begin{aligned}
        R \cdot \psi_1 &= \begin{bmatrix}
            \cos(\theta) & -\sin(\theta) \\
            \sin(\theta) & \cos(\theta)
        \end{bmatrix} \cdot \begin{bmatrix}
            \frac{1}{\sqrt{2}} \\ -\frac{1}{\sqrt{2}}
        \end{bmatrix} \\
        &= \begin{bmatrix}
            \frac{1}{\sqrt{2}}\cos(\theta) + \frac{1}{\sqrt{2}}\sin(\theta) \\
            \frac{1}{\sqrt{2}}\sin(\theta) - \frac{1}{\sqrt{2}}\cos(\theta)
        \end{bmatrix}
    \end{aligned}
\end{equation*}

The probability of successfully distinguishing the quantum state is as follows:

\begin{equation*}
    \begin{aligned}
        P[\hat{b} = b]
        &= P[\hat{b} = b \cap b = 0] + P[\hat{b} = b \cap b = 1] \\
        &= P[\hat{b} = b \mid b = 0] \cdot P[b = 0] 
            + P[\hat{b} = b \mid b = 1] \cdot P[b = 1] \\
        &= \frac{1}{2}(P[\hat{b} = b \mid b = 0] + P[\hat{b} = b \mid b = 1]) \\
        &= \frac{1}{2}(P[R\psi_0 = 0] + P[R\psi_1 = 1]) \\
        &= \frac{1}{2}(\sin^2(\theta) + (
                \frac{1}{\sqrt{2}}\sin(\theta) - \frac{1}{\sqrt{2}}\cos(\theta)
            )^2)
    \end{aligned}
\end{equation*}

To maximize $P[\hat{b} = b]$ with respect to $\theta$, take the derivative:

\begin{equation*}
    \begin{aligned}
        \frac{dP}{d\theta}
        &= \frac{1}{2}\left(
            2\sin\theta\cos\theta
            + 2(
                \frac{1}{\sqrt{2}}\sin\theta - \frac{1}{\sqrt{2}}\cos\theta
            ) \cdot \frac{d}{d\theta}\left\lbrack 
                \frac{1}{\sqrt{2}}\sin\theta - \frac{1}{\sqrt{2}}\cos\theta
            \right\rbrack
        \right) \\
        &= \frac{1}{2}\left(
            \sin{2\theta}
            + 2(
                \frac{1}{\sqrt{2}}\sin\theta - \frac{1}{\sqrt{2}}\cos\theta
            ) \cdot \left(
                \frac{1}{\sqrt{2}}\cos\theta + \frac{1}{\sqrt{2}}\sin\theta
            \right)
        \right) \\
        &= \frac{1}{2}\left(
            \sin{2\theta} -\cos{2\theta}
        \right) \\
        \frac{d^2P}{d\theta^2} &= \cos{\theta} + \sin{\theta}
    \end{aligned}
\end{equation*}

Solving the first derivative yields $\theta = \frac{1}{8}\pi$ or $\theta = \frac{5}{8}\pi$, but only the second solution makes the second derivative negative and is thus a local maximum.

Thus an optimal distinguisher can be built by rotation $R(\frac{5}{8}\pi)$, the maximal probability is:

\begin{equation*}
    \frac{1}{2}\left(
        \sin^2{\frac{5}{8}\pi} + \left(
            \frac{1}{\sqrt{2}}\sin{\frac{5}{8}\pi} - \frac{1}{\sqrt{2}}\cos{\frac{5}{8}\pi}
        \right)^2
    \right)
    \approx 0.85355
\end{equation*}

\end{document}