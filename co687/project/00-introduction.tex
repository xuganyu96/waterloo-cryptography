\section{Introduction}

Trapdoor functions are the foundations of public-key cryptography. The most famous and popular trapdoor functions include the RSA trapdoor (it is easy to compute $m^e$ from $m$ and $e$ but difficult to recover $m$ from $m^e$ and $e$) and discrete log (it is easy to compute $g^x$ from $g, x$ but hard to recover $x$ from $g^x, g$). While these two trapdoors have been proved very successful over their decades of application, they are unfortunately both based on the hardness of integer factorization. Trapdoor functions based on other classes of hard problems have been of research interest since the early days of public-key cryptography, and they have gained increased attention as a result of the discovery of efficient quantum algorithm for integer factorization that threatens to break encryption and signature schemes based on RSA and Diffie-Hellman.

In this project is a survey a particular class of cryptographic trapdoor functions whose construction is based on lattices and whose security is based on the hardness of lattice problems such as the shortest vector problem (SVP) and the closest vector problem (CVP). Some preliminary mathematics surrounding lattice are introduced in chapter 2, followed by some additional discussion of hard lattice problems and known best algorithms to solve them in chapter 3. A lattice-based trapdoor, initially proposed by Goldreich-Goldwasser-Halevi in 1997, is described and its applications discussed in chapter 4. In chapter 5 we will discuss an improvement due to Micciancio that uses Hermite normal form to reduce key sizes while achieving optimal security.