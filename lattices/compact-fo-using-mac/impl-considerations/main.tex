\documentclass{article}
\usepackage[margin=1in,letterpaper]{geometry}
\usepackage{amsmath,amsfonts,amssymb,amsthm}

% For source code
\usepackage{listings}

% Algorithms and pseudocode
\usepackage[linesnumbered,ruled,vlined]{algorithm2e}
\usepackage{algpseudocode}
\usepackage{multicol}

% Common shortcuts
\newcommand{\monospace}{\texttt}
\newcommand{\pke}{\monospace{PKE}}
\newcommand{\keygen}{\monospace{KeyGen}}
\newcommand{\encrypt}{\monospace{E}}
\newcommand{\decrypt}{\monospace{D}}
\newcommand{\kem}{\monospace{KEM}}
\newcommand{\encap}{\monospace{Encap}}
\newcommand{\decap}{\monospace{Decap}}
\newcommand{\etm}{\monospace{EtM}}  % encrypt-then-mac
\newcommand{\mac}{\monospace{MAC}}
\newcommand{\sign}{\monospace{MAC}}
\newcommand{\verify}{\monospace{MAC.Verify}}
\newcommand{\pk}{\monospace{pk}}
\newcommand{\sk}{\monospace{sk}}
\newcommand{\pco}{\monospace{PCO}}
\newcommand{\cvo}{\monospace{CVO}}
\newcommand{\leftsample}{\stackrel{\$}{\leftarrow}}
\newcommand{\llbrack}{[\![}
\newcommand{\rrbrack}{]\!]}
\newcommand{\norm}[1]{\left\lvert #1 \right\rvert}

% Environments: definitions, theorems, propositions, corollaries, lemmas
%    Theorems, propositions, and definitions are numbered within the section
%    Corollaries are numbered within the theorem, though they are rarely used
\newtheorem{definition}{Definition}[section]
\newtheorem{theorem}{Theorem}[section]
\newtheorem*{remark}{Remark}
\newtheorem{corollary}{Corollary}[theorem]
\newtheorem{proposition}{Proposition}[section]
\newtheorem{lemma}{Lemma}[theorem]


\title{Some implementation considerations}
\author{Ganyu (Bruce) Xu (g66xu)}
% \date{June, 2024}

\begin{document}
%%%% TITLE %%%%%
\maketitle

\section{One-time MAC}
The challenge encryption $c^\ast$ in the \monospace{OW-PCVA} game is computed using a uniformly randomly sampled plaintext message $m^\ast \leftsample \mathcal{M}_\pke$, and a $\mac$ key $k_\mac$ derived from said plaintext $m^\ast$. Assuming that $m^\ast$ is not already known (otherwise the \monospace{OW-CPA} security of the underlying $\pke$ is already broken), then $k^\ast_\mac$ is also not known, so the \monospace{OW-PCVA} adversary has no way to evaluate the $\mac$ under $k^\ast_\mac$ offline.

Furthermore, I propose that the \monospace{OW-PCVA} adversary also has no way of issuing signing queries for $k^\ast_\mac$. The adversary can run $\encrypt$ or $\encap$ offline, but the tags generated from $\encrypt$ or $\encap$ will be associated with a chosen message or a random message, which has $\frac{1}{\norm{\mathcal{M}_\pke}}$ probability of being $m^\ast$. In other words, for the challenge $\mac$ key, the adversary will only see one valid message-tag pair, which is the challenge message $(c^\ast, t^\ast)$.

This is the literal definition of a one-time MAC. All many-time MAC are valid one-time MAC, but there are specific one-time MAC constructions that might be even more computationally efficient. \textbf{GMAC and Poly1305 with 256-bit keys and 128-bit tag} are great candidates, as are any Carter-Wegman constructions $t = h(m) \oplus \monospace{PRF}(r, k)$.

We need 256 bits for the $\mac$ key, because checking a $\mac$ key can be done offline (the adversary takes a guess at the key, computes the tag using the guessed key and see if it matches the known tag) and is therefore vulnerable to Grover's algorithm.

On the other hand, a 128-bit tag is not a security concern. This is because for one, the adversary has no way to issuing additional signing queries, so there is no concern for collision. The adversary can use the decapsulation oracle to check whether a forgery $(\hat{c}, \hat{t})$ is valid, but assuming that the $\mac$ is an ideal PRF, the adversary will have no better way of forgery than random guesses at the tag value. In an actual setting, it is easy for the peer to implement some rate limiting such that after only a small number of invalid ciphertext-tag pairs the adversary will simply be blocked.

\section{Choice of MAC}
When thinking about choosing a MAC there are two parameters that we need to care about: the key size and the tag size.

Whether working with Round 3 Kyber \cite{avanzi2019crystals} or ML-KEM \cite{key2023mechanism}, the message is always 256 bits in size, and since the MAC key is derived from the message, the MAC key should also always be 256 bits. A 256-bit MAC key is great since both GCM and Poly1305 accepts 256-bit MAC keys.

Reasoning about the security implication of the tag size is more nuanced.

\subsection{Generic collision resistance}
We can model a MAC has a keyed hash function, which means that for a fixed MAC key, we can argue about the seurity of a MAC using the standard security notions for hash function: pre-image resistance, 2nd pre-image resistance, and collision resistance.

However, there is an important distinction between the security of a MAC as a keyed hash function and the security of a hash function. With MAC as a keyed hash function, assuming that the adversary doesn't already have a way to evaluating this function offline (otherwise forgery is trivial), \textbf{it has to query the signing oracle to evaluate the hash function}. Given an n-bit tag, a generic birthday attack takes $2^{\frac{n}{2}}$ evaluations to find a collision, and the BHT algorithm \cite{brassard1997quantum} takes $2^{\frac{n}{3}}$ evaluations to find a collision, but neither attack is relevant, because "evaluation" means that the message-tag pair is queried from the signing oracle and \textbf{does not count as forgery}.

This means that we only need to care about preimage resistance for our MAC (as a keyed hash function), 

\subsection{Offline versus online attacks}
There should be a distinction between attacks that involve interaction with another party (online) versus attacks that can be done without such interaction (offline). When we measure the security of a scheme by the number of computational steps needed to carry out an attack, we specifically mean an offline attack. \textbf{An online attack with substantial number of interactions can be trivially thwarted} because the other party can implement rate limiting with exponential backoff, so after only a few (think fewer than 10 password tries) invalid interactions, the other party will effectively block the adversary from further interactions, thus stopping the attack in the track.

\textbf{Adversary obtains $(c, t)$}, tampers with $c$ to obtain $c^\prime$ which still correspond with the same plaintext $m$, which means that $t^\prime$ should be created under the same key $(c, t)$ is created.

\bibliographystyle{plain}
\bibliography{./references.bib}
\end{document}