\documentclass{article}
\usepackage[margin=1in,letterpaper]{geometry}
\usepackage{amsmath,amsfonts,amssymb,amsthm}
\usepackage{multicol}

% For source code
\usepackage{listings}

% Algorithms and pseudocode
\usepackage[linesnumbered,ruled,vlined]{algorithm2e}
\usepackage{algpseudocode}

% Common shortcuts
\newcommand{\round}[1]{\lfloor {#1} \rceil}
\newcommand{\Norm}[1]{\Vert {#1} \Vert}
\newcommand{\norm}[1]{\vert {#1} \vert}
\newcommand{\var}[1]{\operatorname{Var}[{#1}]}
\newcommand{\leftsample}{\overset{{\scriptscriptstyle\$}}{\leftarrow}}
\newcommand{\keygen}{\operatorname{KeyGen}}
\newcommand{\pk}{\operatorname{pk}}
\newcommand{\sk}{\operatorname{sk}}
\newcommand{\pco}{\operatorname{PCO}}
\newcommand{\cvo}{\operatorname{CVO}}
\newcommand{\coin}{\operatorname{COIN}}
\newcommand{\llbrack}{[\![}
\newcommand{\rrbrack}{]\!]}
\newlength{\wdth}
\newcommand{\strike}[1]{\settowidth{\wdth}{#1}\rlap{\rule[.5ex]{\wdth}{.4pt}}#1}

% Environments: definitions, theorems, propositions, corollaries, lemmas
%    Theorems, propositions, and definitions are numbered within the section
%    Corollaries are numbered within the theorem, though they are rarely used
\newtheorem{definition}{Definition}[section]
\newtheorem{theorem}{Theorem}[section]
\newtheorem*{remark}{Remark}
\newtheorem{corollary}{Corollary}[theorem]
\newtheorem{proposition}{Proposition}[section]
\newtheorem{lemma}{Lemma}[theorem]


\title{Some clarification on security definitions}
\author{Ganyu (Bruce) Xu (g66xu)}
\date{Spring, 2024}

\begin{document}
%%%% TITLE %%%%%
\maketitle

The confusion about what various security definition means seems to be caused by some inconsistencies between the popular textbooks and papers. I consulted two textbooks and found that indeed their security definitions are meaningfully different from what I am familiar with.

\section{Textbook definitions}
``A graduate course in applied cryptography''\cite{boneh2020graduate} introduced the concept of \textbf{CPA security} in section 5.3 (page 181):

\begin{definition}[CPA security from Boneh and Shoup]
    For a given cipher $\mathcal{E} = (E, D)$ defined over $(\mathcal{K}, \mathcal{M}, \mathcal{C})$ and for a given adversary $\mathcal{A}$, we define experiments $b$ for $b = 0, 1$:

    \begin{enumerate}
        \item The challenge selects $k \leftsample \mathcal{K}$
        \item The adversary submits a sequence of queries to the challenge. For $i = 1, 2, \ldots$, the $i$-th query is a pair of messages $m_{i, 0}, m_{i, 1}$ of the same length. The challenger computes $c_i \leftarrow E(k, m_{i, b})$ and return $c_i$ to the adversary
        \item The adversary outputs a bit $\hat{b} \in \{0, 1\}$
    \end{enumerate}

    Let $W_b$ denote the event that $\mathcal{A}$ outputs 1 in experiment $b$. We define $\mathcal{A}$'s advantage with respect to $\mathcal{E}$ to be:

    \begin{equation*}
        \operatorname{CPAadv}[A, \mathcal{E}] = \norm{
            P[W_0] - P[W_1]
        }
    \end{equation*}

    A cipher $\mathcal{E}$ is called \textbf{semantically secure against chosen plaintext attack}, or simply \textbf{CPA secure} if for all efficient adversaries, $\operatorname{CPAadv}$ is negligible.
\end{definition}


``Introduction to modern cryptography''\cite{katz2007introduction} also introduced the concept of CPA security in the context of an adversarial game:

\begin{definition}[CPA security from Katz and Lindell]
    We first define an experiment for any encryption scheme, any adversary, and any value $\lambda$ for the security parameter:

    \begin{enumerate}
        \item A random key is generated
        \item The adversary is given oracle access to the encryption routine and outputs a pair of messages of the same length
        \item A random bit is chosen and a ciphertext computed and given to the adversary
        \item The adversary outputs a bit
        \item The adversary wins if the output bit is equal to the random bit
    \end{enumerate}

    An encryption scheme has \textbf{indistinguishable encryptions under a chosen-plaintext attack}, or is \textbf{CPA secure}, if for all PPT adversaries there exists a negligible function $\operatorname{negl}$ such that

    \begin{equation*}
        P[\hat{b} = b^\ast] \leq \frac{1}{2} + \operatorname{negl}(\lambda)
    \end{equation*}
\end{definition}

\section{Security definition in research paper}
In ``A modular analysis of the Fujisaki-Okamoto transformation''\cite{hofheinz2017modular} by Hofheinz et al, the security definitions are as follows:

\begin{definition}[OW-ATK]
    Let $\operatorname{PKE} = (\operatorname{Gen}, \operatorname{Enc}, \operatorname{Dec})$ be a public-key encryption scheme with message space $\mathcal{M}$. For $\operatorname{ATK} \in \{\operatorname{CPA}, \operatorname{PCA}, \operatorname{VA}, \operatorname{PCVA}\}$ we define OW-ATK game, where

    \begin{equation*}
        \mathcal{O}_{\operatorname{ATK}} = \begin{cases}
            - & \operatorname{ATK} = \operatorname{CPA} \\
            \operatorname{PCO} & \operatorname{ATK} = \operatorname{CPA} \\
            \operatorname{CVO} & \operatorname{ATK} = \operatorname{VA} \\
            \operatorname{PCO}, \operatorname{CVO} & \operatorname{ATK} = \operatorname{PCVA}
        \end{cases}
    \end{equation*}

    \begin{multicols}{2}
        \begin{algorithm}[H]
            \caption{$\operatorname{OW-ATK}$ game}
            $(\operatorname{pk}, \operatorname{sk}) \leftarrow \operatorname{Gen}()$\;
            $m^\ast \leftsample \mathcal{M}$\;
            $c^\ast \leftarrow E(\operatorname{pk}, m^\ast)$\;
            $\hat{m} \leftarrow \mathcal{A}^{\mathcal{O}_{\operatorname{ATK}}}(\operatorname{pk}, c^\ast)$\;
            \Return{$\llbrack \hat{m} = m^\ast \rrbrack$}
        \end{algorithm}

        \begin{algorithm}[H]
            \caption{$\pco(m \in \mathcal{M}, c)$}
            \Return{$\llbrack D(\sk, c) = m \rrbrack$}
        \end{algorithm}

        \begin{algorithm}[H]
            \caption{$\cvo(c \neq c^\ast)$}
            \Return{$\llbrack D(\sk, c) \in \mathcal{M} \rrbrack$}
        \end{algorithm}
    \end{multicols}
\end{definition}

\section{Conclusion}
In the textbooks, ``\textbf{CPA secure}'' really means ``\textbf{IND-CPA} secure''. On the other hand, in research paper, the security definition is always explicitly spelled with both the goal (to break one-wayness or to break indistinguishability) and the adversary's capabilities (access to some specified set of oracles). My guess is that we got confused last week because in public-key cryptography, CPA is a rather meaningless notion because the adversary has the public key, so ``one-wayness'' automatically implies ``one-way security under CPA''.

For an example, here is textbook RSA:

\begin{multicols}{2}
    \begin{algorithm}[H]
        \caption{RSA KeyGen}
        $p, q \leftsample \operatorname{PrimeGen}()$\;
        $N \leftarrow p \cdot q$\;
        $\phi \leftarrow (p-1)\cdot(q-1)$\;
        $e \leftarrow 3$\;
        $d \leftarrow e^{-1} \mod \phi$\;
        \Return{
            $\pk = (N, e)$, $\sk = d$
        }
    \end{algorithm}

    \begin{algorithm}[H]
        \caption{Encryption $E(\pk, m)$}
        \Return{
            $m^e \mod N$
        }
    \end{algorithm}

    \begin{algorithm}[H]
        \caption{Decryption $D(\sk, c)$}
        \Return{
            $c^d \mod N$
        }
    \end{algorithm}
\end{multicols}

From the textbooks, \textbf{textbook RSA achieves one-wayness but is not CPA secure}, because its encryption is deterministic. On the other hand, using explicit game definitions, we say that RSA is OW-CPA secure but not IND-CPA secure.

OW-CPA and IND-CPA security are commonly accepted standard security notions in. Both Hofheinz\cite{hofheinz2017modular} and the Kyber team\cite{bos2018crystals} make use of OW-CPA and IND-CPA in their papers. In any case, since I will introduce non-standard security notions, I will explicitly spell out the security games, including the adversary's goal and the adversary's capabilities, so there should be no confusion about the meaning of any terms.

\bibliographystyle{plain}
\bibliography{./references.bib}

\end{document}